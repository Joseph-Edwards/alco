% generated by GAPDoc2LaTeX from XML source (Frank Luebeck)
\documentclass[a4paper,11pt]{report}

\usepackage[top=37mm,bottom=37mm,left=27mm,right=27mm]{geometry}
\sloppy
\pagestyle{myheadings}
\usepackage{amssymb}
\usepackage[latin1]{inputenc}
\usepackage{makeidx}
\makeindex
\usepackage{color}
\definecolor{FireBrick}{rgb}{0.5812,0.0074,0.0083}
\definecolor{RoyalBlue}{rgb}{0.0236,0.0894,0.6179}
\definecolor{RoyalGreen}{rgb}{0.0236,0.6179,0.0894}
\definecolor{RoyalRed}{rgb}{0.6179,0.0236,0.0894}
\definecolor{LightBlue}{rgb}{0.8544,0.9511,1.0000}
\definecolor{Black}{rgb}{0.0,0.0,0.0}

\definecolor{linkColor}{rgb}{0.0,0.0,0.554}
\definecolor{citeColor}{rgb}{0.0,0.0,0.554}
\definecolor{fileColor}{rgb}{0.0,0.0,0.554}
\definecolor{urlColor}{rgb}{0.0,0.0,0.554}
\definecolor{promptColor}{rgb}{0.0,0.0,0.589}
\definecolor{brkpromptColor}{rgb}{0.589,0.0,0.0}
\definecolor{gapinputColor}{rgb}{0.589,0.0,0.0}
\definecolor{gapoutputColor}{rgb}{0.0,0.0,0.0}

%%  for a long time these were red and blue by default,
%%  now black, but keep variables to overwrite
\definecolor{FuncColor}{rgb}{0.0,0.0,0.0}
%% strange name because of pdflatex bug:
\definecolor{Chapter }{rgb}{0.0,0.0,0.0}
\definecolor{DarkOlive}{rgb}{0.1047,0.2412,0.0064}


\usepackage{fancyvrb}

\usepackage{mathptmx,helvet}
\usepackage[T1]{fontenc}
\usepackage{textcomp}


\usepackage[
            pdftex=true,
            bookmarks=true,        
            a4paper=true,
            pdftitle={Written with GAPDoc},
            pdfcreator={LaTeX with hyperref package / GAPDoc},
            colorlinks=true,
            backref=page,
            breaklinks=true,
            linkcolor=linkColor,
            citecolor=citeColor,
            filecolor=fileColor,
            urlcolor=urlColor,
            pdfpagemode={UseNone}, 
           ]{hyperref}

\newcommand{\maintitlesize}{\fontsize{50}{55}\selectfont}

% write page numbers to a .pnr log file for online help
\newwrite\pagenrlog
\immediate\openout\pagenrlog =\jobname.pnr
\immediate\write\pagenrlog{PAGENRS := [}
\newcommand{\logpage}[1]{\protect\write\pagenrlog{#1, \thepage,}}
%% were never documented, give conflicts with some additional packages

\newcommand{\GAP}{\textsf{GAP}}

%% nicer description environments, allows long labels
\usepackage{enumitem}
\setdescription{style=nextline}

%% depth of toc
\setcounter{tocdepth}{1}





%% command for ColorPrompt style examples
\newcommand{\gapprompt}[1]{\color{promptColor}{\bfseries #1}}
\newcommand{\gapbrkprompt}[1]{\color{brkpromptColor}{\bfseries #1}}
\newcommand{\gapinput}[1]{\color{gapinputColor}{#1}}


\begin{document}

\logpage{[ 0, 0, 0 ]}
\begin{titlepage}
\mbox{}\vfill

\begin{center}{\maintitlesize \textbf{The \textsf{ALCO} Package\mbox{}}}\\
\vfill

\hypersetup{pdftitle=The \textsf{ALCO} Package}
\markright{\scriptsize \mbox{}\hfill The \textsf{ALCO} Package \hfill\mbox{}}
{\Huge Version 0.1\mbox{}}\\[1cm]
\mbox{}\\[2cm]
{\Large \textbf{Benjamin Nasmith  \mbox{}}}\\
\hypersetup{pdfauthor=Benjamin Nasmith  }
\end{center}\vfill

\mbox{}\\
{\mbox{}\\
\small \noindent \textbf{Benjamin Nasmith  }  Email: \href{mailto://bnasmith@proton.me} {\texttt{bnasmith@proton.me}}}\\
\end{titlepage}

\newpage\setcounter{page}{2}
{\small 
\section*{Copyright}
\logpage{[ 0, 0, 1 ]}
{\copyright} 2023 The Author. 

 

 \mbox{}}\\[1cm]
\newpage

\def\contentsname{Contents\logpage{[ 0, 0, 2 ]}}

\tableofcontents
\newpage

 
\chapter{\textcolor{Chapter }{Introduction}}\logpage{[ 1, 0, 0 ]}
\hyperdef{L}{X7DFB63A97E67C0A1}{}
{
  The \textsf{ALCO} package provides tools for algebraic combinatorics, most of which was written
for \textsf{GAP} during the author's Ph.D. program \cite{nasmith_tight_2023}. This package provides convenient constructions in \textsf{GAP} of octonion algebras, Jordan algebras, and certain important integer subrings
of those algebras. It also provides tools to compute the parameters of
t-designs in spherical and projective spaces (modeled as manifolds of
primitive idempotent elements in a simple Euclidean Jordan algebra). Finally,
this package provides tools to explore octonion lattice constructions,
including octonion Leech lattices. }

 
\chapter{\textcolor{Chapter }{Octonions}}\logpage{[ 2, 0, 0 ]}
\hyperdef{L}{X7E7EE82D811283C0}{}
{
  \textsf{GAP} contains limited built-in functionality for constructing and manipulating
octonions. The built-in \texttt{OctaveAlgebra} function constructs the split-octonion algebra over some field. The \textsf{ALCO} package provides constructions of non-split octonion algebras in various
bases. 
\section{\textcolor{Chapter }{Octonion Algebras}}\label{sec:octalg}
\logpage{[ 2, 1, 0 ]}
\hyperdef{L}{X7833529F8000FCAD}{}
{
  
\subsection{\textcolor{Chapter }{Octonion Filters}}\logpage{[ 2, 1, 1 ]}
\hyperdef{L}{X81A45FA7806BF5AC}{}
{
\noindent\textcolor{FuncColor}{$\triangleright$\enspace\texttt{IsOctonion\index{IsOctonion@\texttt{IsOctonion}}
\label{IsOctonion}
}\hfill{\scriptsize (filter)}}\\
\noindent\textcolor{FuncColor}{$\triangleright$\enspace\texttt{IsOctonionArithmeticElement\index{IsOctonionArithmeticElement@\texttt{IsOctonionArithmeticElement}}
\label{IsOctonionArithmeticElement}
}\hfill{\scriptsize (filter)}}\\
\noindent\textcolor{FuncColor}{$\triangleright$\enspace\texttt{IsOctonionCollection\index{IsOctonionCollection@\texttt{IsOctonionCollection}}
\label{IsOctonionCollection}
}\hfill{\scriptsize (filter)}}\\
\noindent\textcolor{FuncColor}{$\triangleright$\enspace\texttt{IsOctonionAlgebra\index{IsOctonionAlgebra@\texttt{IsOctonionAlgebra}}
\label{IsOctonionAlgebra}
}\hfill{\scriptsize (filter)}}\\


These filters determine whether an element is an octonion, an octonion
arithmetic element, and octonion collection, or an octonion algebra.}

 

\subsection{\textcolor{Chapter }{OctonionAlgebra}}
\logpage{[ 2, 1, 2 ]}\nobreak
\hyperdef{L}{X78767B4A7F44F77D}{}
{\noindent\textcolor{FuncColor}{$\triangleright$\enspace\texttt{OctonionAlgebra({\mdseries\slshape F})\index{OctonionAlgebra@\texttt{OctonionAlgebra}}
\label{OctonionAlgebra}
}\hfill{\scriptsize (function)}}\\


 Returns an octonion algebra over field \mbox{\texttt{\mdseries\slshape F}} in a standard orthonormal basis $\{e_{i}, i = 1,...,8\}$ such that $1 = e_8$ is the identity element and $e_{i} = e_{i+1}e_{i+3} = - e_{i+3}e_{i+1}$ for $i = 1,...,7$, with indices evaluated modulo 7. 
\begin{Verbatim}[commandchars=!@|,fontsize=\small,frame=single,label=Example]
  !gapprompt@gap>| !gapinput@O := OctonionAlgebra(Rationals); e := Basis(O);;|
  <algebra of dimension 8 over Rationals>
  !gapprompt@gap>| !gapinput@LeftActingDomain(O);|
  Rationals
  !gapprompt@gap>| !gapinput@AsList(e);|
  [ e1, e2, e3, e4, e5, e6, e7, e8 ]
  !gapprompt@gap>| !gapinput@One(O);|
  e8
  !gapprompt@gap>| !gapinput@e[1]*e[2];|
  e4
  !gapprompt@gap>| !gapinput@e[2]*e[1];|
  (-1)*e4
  !gapprompt@gap>| !gapinput@Derivations(Basis(O)); SemiSimpleType(last);|
  <Lie algebra of dimension 14 over Rationals>
  "G2"
\end{Verbatim}
 }

 

\subsection{\textcolor{Chapter }{OctonionArithmetic}}
\logpage{[ 2, 1, 3 ]}\nobreak
\hyperdef{L}{X7CDCBD9B85787CA1}{}
{\noindent\textcolor{FuncColor}{$\triangleright$\enspace\texttt{OctonionArithmetic({\mdseries\slshape R})\index{OctonionArithmetic@\texttt{OctonionArithmetic}}
\label{OctonionArithmetic}
}\hfill{\scriptsize (function)}}\\


 Returns an octonion algebra over \mbox{\texttt{\mdseries\slshape R}}, for \mbox{\texttt{\mdseries\slshape R}} a field or \texttt{\mbox{\texttt{\mdseries\slshape R}} = Integers}, in a basis with the geometry of the simple roots of $E_8$ such that {\ensuremath{\mathbb Z}}-linear combinations of the basis vectors
form an octonion arithmetic. 
\begin{Verbatim}[commandchars=!@|,fontsize=\small,frame=single,label=Example]
  !gapprompt@gap>| !gapinput@A := OctonionArithmetic(Integers); a := Basis(A);;|
  <algebra of dimension 8 over Integers>
  !gapprompt@gap>| !gapinput@LeftActingDomain(A);|
  Integers
  !gapprompt@gap>| !gapinput@AsList(a);|
  [ a1, a2, a3, a4, a5, a6, a7, a8 ]
  !gapprompt@gap>| !gapinput@One(A);|
  (-2)*a1+(-3)*a2+(-4)*a3+(-6)*a4+(-5)*a5+(-4)*a6+(-3)*a7+(-2)*a8
  !gapprompt@gap>| !gapinput@List(a{[1..7]}, x -> x^2 = - One(A));|
  [ true, true, true, true, true, true, true ]
  !gapprompt@gap>| !gapinput@Order(a[8]);|
  3
  !gapprompt@gap>| !gapinput@Random(A)*Random(A) in A;|
  true
\end{Verbatim}
 }

 

\subsection{\textcolor{Chapter }{Oct}}
\logpage{[ 2, 1, 4 ]}\nobreak
\hyperdef{L}{X7CD8436C85792B3B}{}
{\noindent\textcolor{FuncColor}{$\triangleright$\enspace\texttt{Oct\index{Oct@\texttt{Oct}}
\label{Oct}
}\hfill{\scriptsize (global variable)}}\\


 The \textsf{ALCO} package loads an instance of \texttt{OctonionAlgebra} (\ref{OctonionAlgebra}) over {\ensuremath{\mathbb Q}} as \texttt{Oct}. 
\begin{Verbatim}[commandchars=!@|,fontsize=\small,frame=single,label=Example]
  !gapprompt@gap>| !gapinput@Oct;|
  <algebra of dimension 8 over Rationals>
\end{Verbatim}
 }

 

\subsection{\textcolor{Chapter }{OctonionE8basis}}
\logpage{[ 2, 1, 5 ]}\nobreak
\hyperdef{L}{X7E4DEB1E7C7F2C1D}{}
{\noindent\textcolor{FuncColor}{$\triangleright$\enspace\texttt{OctonionE8basis\index{OctonionE8basis@\texttt{OctonionE8basis}}
\label{OctonionE8basis}
}\hfill{\scriptsize (global variable)}}\\


 The \textsf{ALCO} package also loads a basis for \texttt{Oct} (\ref{Oct}) which also serves as the {\ensuremath{\mathbb Z}}-span of an octonion
arithmetic. This basis also serves as the basis vectors for the \texttt{OctonionArithmetic} (\ref{OctonionArithmetic}) algebra. 
\begin{Verbatim}[commandchars=!@|,fontsize=\small,frame=single,label=Example]
  !gapprompt@gap>| !gapinput@2*BasisVectors(OctonionE8basis);|
  [ (-1)*e1+e5+e6+e7, (-1)*e1+(-1)*e2+(-1)*e4+(-1)*e7, e2+e3+(-1)*e5+(-1)*e7,
    e1+(-1)*e3+e4+e5, (-1)*e2+e3+(-1)*e5+e7, e2+(-1)*e4+e5+(-1)*e6,
    (-1)*e1+(-1)*e3+e4+(-1)*e5, e1+(-1)*e4+e6+(-1)*e8 ]
\end{Verbatim}
 }

 

\subsection{\textcolor{Chapter }{\texttt{\symbol{92}}mod}}
\logpage{[ 2, 1, 6 ]}\nobreak
\hyperdef{L}{X87F02BDC8090F3EE}{}
{\noindent\textcolor{FuncColor}{$\triangleright$\enspace\texttt{\texttt{\symbol{92}}mod({\mdseries\slshape x, n})\index{\texttt{\symbol{92}}mod@\texttt{\texttt{\symbol{92}}mod}}
\label{bSlashmod}
}\hfill{\scriptsize (function)}}\\


 For \mbox{\texttt{\mdseries\slshape x}} an octonion arithmetic element (namely an element of algebra \texttt{OctonionArithmetic(\mbox{\texttt{\mdseries\slshape F}})}), and \mbox{\texttt{\mdseries\slshape n}} and integer, the expression \texttt{\mbox{\texttt{\mdseries\slshape x}} mod \mbox{\texttt{\mdseries\slshape n}}} returns the octonion where each of the coefficients in the arithmetic
canonical basis have been evaluated modulo \mbox{\texttt{\mdseries\slshape n}}. 
\begin{Verbatim}[commandchars=!@|,fontsize=\small,frame=single,label=Example]
  !gapprompt@gap>| !gapinput@A := OctonionArithmetic(Integers);|
  <algebra of dimension 8 over Integers>
  !gapprompt@gap>| !gapinput@x := Random(A);|
  (-2)*a2+(3)*a3+a4+(-2)*a6+(-1)*a7+(-1)*a8
  !gapprompt@gap>| !gapinput@x mod 2;|
  a3+a4+a7+a8
  !gapprompt@gap>| !gapinput@\mod(x,2);|
  a3+a4+a7+a8
\end{Verbatim}
 }

 }

 
\section{\textcolor{Chapter }{Properties of Octonions}}\label{sec:octattr}
\logpage{[ 2, 2, 0 ]}
\hyperdef{L}{X86E4523081C49806}{}
{
  

\subsection{\textcolor{Chapter }{Norm (Octonions)}}
\logpage{[ 2, 2, 1 ]}\nobreak
\hyperdef{L}{X7CEAB1C67B22DA7E}{}
{\noindent\textcolor{FuncColor}{$\triangleright$\enspace\texttt{Norm({\mdseries\slshape x})\index{Norm@\texttt{Norm}!Octonions}
\label{Norm:Octonions}
}\hfill{\scriptsize (method)}}\\


Returns the norm of octonion \mbox{\texttt{\mdseries\slshape x}}. Recall that an octonion algebra satisfies the composition property $N(xy) = N(x)N(y)$. 
\begin{Verbatim}[commandchars=!@|,fontsize=\small,frame=single,label=Example]
  !gapprompt@gap>| !gapinput@List(Basis(Oct), x -> Norm(x));|
  [ 1, 1, 1, 1, 1, 1, 1, 1 ]
  !gapprompt@gap>| !gapinput@x := Random(Oct);; y := Random(Oct);;|
  !gapprompt@gap>| !gapinput@Norm(x*y) = Norm(x)*Norm(y);|
  true
\end{Verbatim}
 }

 

\subsection{\textcolor{Chapter }{Trace (Octonions)}}
\logpage{[ 2, 2, 2 ]}\nobreak
\hyperdef{L}{X8794715F82DE210B}{}
{\noindent\textcolor{FuncColor}{$\triangleright$\enspace\texttt{Trace({\mdseries\slshape x})\index{Trace@\texttt{Trace}!Octonions}
\label{Trace:Octonions}
}\hfill{\scriptsize (method)}}\\


Returns the trace of octonion \mbox{\texttt{\mdseries\slshape x}}. 
\begin{Verbatim}[commandchars=!@|,fontsize=\small,frame=single,label=Example]
  !gapprompt@gap>| !gapinput@List(Basis(Oct), x -> Trace(x));|
  [ 0, 0, 0, 0, 0, 0, 0, 2 ]
\end{Verbatim}
 }

 

\subsection{\textcolor{Chapter }{GramMatrix (GramMatrixOctonion)}}
\logpage{[ 2, 2, 3 ]}\nobreak
\hyperdef{L}{X847240C08217CAC7}{}
{\noindent\textcolor{FuncColor}{$\triangleright$\enspace\texttt{GramMatrix({\mdseries\slshape O})\index{GramMatrix@\texttt{GramMatrix}!GramMatrixOctonion}
\label{GramMatrix:GramMatrixOctonion}
}\hfill{\scriptsize (attribute)}}\\


Returns the Gram matrix on the basis of octonion algebra (or arithmetic) \mbox{\texttt{\mdseries\slshape O}} on the basis given by inner product $(x,y) = N(x+y) - N(x) - N(y)$. Of note, the Gram matrix of octonion arithmetic \texttt{A} shown below is the Gram matrix of an $E_8$ unimodular lattice. 
\begin{Verbatim}[commandchars=!@|,fontsize=\small,frame=single,label=Example]
  !gapprompt@gap>| !gapinput@O := OctonionAlgebra(Rationals); Display(GramMatrix(O));|
  <algebra of dimension 8 over Rationals>
  [ [  2,  0,  0,  0,  0,  0,  0,  0 ],
    [  0,  2,  0,  0,  0,  0,  0,  0 ],
    [  0,  0,  2,  0,  0,  0,  0,  0 ],
    [  0,  0,  0,  2,  0,  0,  0,  0 ],
    [  0,  0,  0,  0,  2,  0,  0,  0 ],
    [  0,  0,  0,  0,  0,  2,  0,  0 ],
    [  0,  0,  0,  0,  0,  0,  2,  0 ],
    [  0,  0,  0,  0,  0,  0,  0,  2 ] ]
  !gapprompt@gap>| !gapinput@A := OctonionArithmetic(Rationals); Display(GramMatrix(A));|
  <algebra of dimension 8 over Rationals>
  [ [   2,   0,  -1,   0,   0,   0,   0,   0 ],
    [   0,   2,   0,  -1,   0,   0,   0,   0 ],
    [  -1,   0,   2,  -1,   0,   0,   0,   0 ],
    [   0,  -1,  -1,   2,  -1,   0,   0,   0 ],
    [   0,   0,   0,  -1,   2,  -1,   0,   0 ],
    [   0,   0,   0,   0,  -1,   2,  -1,   0 ],
    [   0,   0,   0,   0,   0,  -1,   2,  -1 ],
    [   0,   0,   0,   0,   0,   0,  -1,   2 ] ]
\end{Verbatim}
 }

 

\subsection{\textcolor{Chapter }{ComplexConjugate (Octonions)}}
\logpage{[ 2, 2, 4 ]}\nobreak
\hyperdef{L}{X7DA1C9FC867AE862}{}
{\noindent\textcolor{FuncColor}{$\triangleright$\enspace\texttt{ComplexConjugate({\mdseries\slshape x})\index{ComplexConjugate@\texttt{ComplexConjugate}!Octonions}
\label{ComplexConjugate:Octonions}
}\hfill{\scriptsize (method)}}\\


Returns the octonion conjugate of octonion \mbox{\texttt{\mdseries\slshape x}}, defined by \texttt{One(x)*Trace(x) - x}. }

 

\subsection{\textcolor{Chapter }{RealPart (Octonions)}}
\logpage{[ 2, 2, 5 ]}\nobreak
\hyperdef{L}{X7FCF154F7BD4E4ED}{}
{\noindent\textcolor{FuncColor}{$\triangleright$\enspace\texttt{RealPart({\mdseries\slshape x})\index{RealPart@\texttt{RealPart}!Octonions}
\label{RealPart:Octonions}
}\hfill{\scriptsize (method)}}\\


Returns the real component of octonion \mbox{\texttt{\mdseries\slshape x}}, defined by \texttt{(1/2)*One(x)*Trace(x)}. }

 

\subsection{\textcolor{Chapter }{ImaginaryPart (Octonions)}}
\logpage{[ 2, 2, 6 ]}\nobreak
\hyperdef{L}{X7A78DC0582712E62}{}
{\noindent\textcolor{FuncColor}{$\triangleright$\enspace\texttt{ImaginaryPart({\mdseries\slshape x})\index{ImaginaryPart@\texttt{ImaginaryPart}!Octonions}
\label{ImaginaryPart:Octonions}
}\hfill{\scriptsize (method)}}\\


Returns the imaginary component of octonion \mbox{\texttt{\mdseries\slshape x}}, defined by \texttt{x - RealPart(x)}. }

 }

 
\section{\textcolor{Chapter }{Other Octonion Tools}}\logpage{[ 2, 3, 0 ]}
\hyperdef{L}{X80488CD07C9B9BD7}{}
{
  

\subsection{\textcolor{Chapter }{OctonionToRealVector}}
\logpage{[ 2, 3, 1 ]}\nobreak
\hyperdef{L}{X83F4932F87EE601F}{}
{\noindent\textcolor{FuncColor}{$\triangleright$\enspace\texttt{OctonionToRealVector({\mdseries\slshape Basis, x})\index{OctonionToRealVector@\texttt{OctonionToRealVector}}
\label{OctonionToRealVector}
}\hfill{\scriptsize (function)}}\\


 Let \mbox{\texttt{\mdseries\slshape x}} be an octonion vector of the form $x = (x_1, x_2, ..., x_n)$, for $x_i$ octonion valued coefficients. Let \mbox{\texttt{\mdseries\slshape Basis}} be a basis for the octonion algebra containing coefficients $x_i$. This function returns a vector \mbox{\texttt{\mdseries\slshape y}} of length $8n$ containing the concatenation of the coefficients of $x_i$ in the octonion basis given by \mbox{\texttt{\mdseries\slshape Basis}}. }

 

\subsection{\textcolor{Chapter }{RealToOctonionVector}}
\logpage{[ 2, 3, 2 ]}\nobreak
\hyperdef{L}{X8433053479E80AEA}{}
{\noindent\textcolor{FuncColor}{$\triangleright$\enspace\texttt{RealToOctonionVector({\mdseries\slshape Basis, y})\index{RealToOctonionVector@\texttt{RealToOctonionVector}}
\label{RealToOctonionVector}
}\hfill{\scriptsize (function)}}\\


 This function is the is the inverse operation to \texttt{OctonionToRealVector} (\ref{OctonionToRealVector}). 
\begin{Verbatim}[commandchars=!@|,fontsize=\small,frame=single,label=Example]
  !gapprompt@gap>| !gapinput@A := OctonionArithmetic(Integers);|
  <algebra of dimension 8 over Integers>
  !gapprompt@gap>| !gapinput@a := Basis(A);; AsList(a);|
  [ a1, a2, a3, a4, a5, a6, a7, a8 ]
  !gapprompt@gap>| !gapinput@x := List([1..3], n -> Random(A));|
  [ (-1)*a1+(-1)*a2+(-1)*a3+a4+(-1)*a5+a6+(-2)*a7+(-1)*a8, (-2)*a1+(-1)*a3+(2)*a4+(-2)*a5+(
      -1)*a6+(2)*a7+(-3)*a8, (-1)*a1+(3)*a2+(-2)*a4+a5+(-4)*a6+a8 ]
  !gapprompt@gap>| !gapinput@OctonionToRealVector(a, x);|
  [ -1, -1, -1, 1, -1, 1, -2, -1, -2, 0, -1, 2, -2, -1, 2, -3, -1, 3, 0, -2, 1, -4, 0, 1 ]
  !gapprompt@gap>| !gapinput@RealToOctonionVector(a,last) = last2;|
  true
\end{Verbatim}
 }

 }

 
\section{\textcolor{Chapter }{Quaternion and Icosian Tools}}\logpage{[ 2, 4, 0 ]}
\hyperdef{L}{X84630FE7831431CE}{}
{
  

\subsection{\textcolor{Chapter }{Norm (Quaternions)}}
\logpage{[ 2, 4, 1 ]}\nobreak
\hyperdef{L}{X7F1D2B237E4AF7A6}{}
{\noindent\textcolor{FuncColor}{$\triangleright$\enspace\texttt{Norm({\mdseries\slshape x})\index{Norm@\texttt{Norm}!Quaternions}
\label{Norm:Quaternions}
}\hfill{\scriptsize (method)}}\\


Returns the norm of quaternion \mbox{\texttt{\mdseries\slshape x}}. Recall that a quaternion algebra satisfies the composition property $N(xy) = N(x)N(y)$. 
\begin{Verbatim}[commandchars=!@|,fontsize=\small,frame=single,label=Example]
  !gapprompt@gap>| !gapinput@H := QuaternionAlgebra(Rationals); AsList(Basis(H));|
  <algebra-with-one of dimension 4 over Rationals>
  [ e, i, j, k ]
  !gapprompt@gap>| !gapinput@List(Basis(H), x -> Norm(x));|
  [ 1, 1, 1, 1 ]
  !gapprompt@gap>| !gapinput@x := Random(H);; y := Random(H);; Norm(x*y) = Norm(x)*Norm(y);|
  true
\end{Verbatim}
 }

 

\subsection{\textcolor{Chapter }{Trace (Quaternions)}}
\logpage{[ 2, 4, 2 ]}\nobreak
\hyperdef{L}{X855FA7867B9D0A9E}{}
{\noindent\textcolor{FuncColor}{$\triangleright$\enspace\texttt{Trace({\mdseries\slshape x})\index{Trace@\texttt{Trace}!Quaternions}
\label{Trace:Quaternions}
}\hfill{\scriptsize (method)}}\\


Returns the trace of quaternion \mbox{\texttt{\mdseries\slshape x}}. 
\begin{Verbatim}[commandchars=!@|,fontsize=\small,frame=single,label=Example]
  !gapprompt@gap>| !gapinput@H := QuaternionAlgebra(Rationals); AsList(Basis(H));|
  <algebra-with-one of dimension 4 over Rationals>
  [ e, i, j, k ]
  !gapprompt@gap>| !gapinput@List(Basis(H), x -> Trace(x));|
  [ 2, 0, 0, 0 ]
\end{Verbatim}
 }

 

\subsection{\textcolor{Chapter }{ComplexConjugate (Quaternions)}}
\logpage{[ 2, 4, 3 ]}\nobreak
\hyperdef{L}{X864255E37BDB3972}{}
{\noindent\textcolor{FuncColor}{$\triangleright$\enspace\texttt{ComplexConjugate({\mdseries\slshape x})\index{ComplexConjugate@\texttt{ComplexConjugate}!Quaternions}
\label{ComplexConjugate:Quaternions}
}\hfill{\scriptsize (method)}}\\


Returns the quaternion conjugate of quaternion \mbox{\texttt{\mdseries\slshape x}}, defined by \texttt{One(x)*Trace(x) - x}.}

 

\subsection{\textcolor{Chapter }{RealPart (Quaternions)}}
\logpage{[ 2, 4, 4 ]}\nobreak
\hyperdef{L}{X81B601A1823F31F0}{}
{\noindent\textcolor{FuncColor}{$\triangleright$\enspace\texttt{RealPart({\mdseries\slshape x})\index{RealPart@\texttt{RealPart}!Quaternions}
\label{RealPart:Quaternions}
}\hfill{\scriptsize (method)}}\\


Returns the real component of quaternion \mbox{\texttt{\mdseries\slshape x}}, defined by \texttt{(1/2)*One(x)*Trace(x)}. }

 

\subsection{\textcolor{Chapter }{ImaginaryPart (Quaternions)}}
\logpage{[ 2, 4, 5 ]}\nobreak
\hyperdef{L}{X7DC868757B5A223C}{}
{\noindent\textcolor{FuncColor}{$\triangleright$\enspace\texttt{ImaginaryPart({\mdseries\slshape x})\index{ImaginaryPart@\texttt{ImaginaryPart}!Quaternions}
\label{ImaginaryPart:Quaternions}
}\hfill{\scriptsize (method)}}\\


Returns the imaginary component of quaternion \mbox{\texttt{\mdseries\slshape x}}, defined by \texttt{x - RealPart(x)}. The \textsf{ALCO} redefines the built in method \texttt{ImaginaryPart} acting on quaternions in order to harmonize with the definition given above
acting on octonions. The most important feature is that for a quaternion or an
octonion, we have \texttt{x = RealPart(x) + ImaginaryPart(x)}. 
\begin{Verbatim}[commandchars=!@|,fontsize=\small,frame=single,label=Example]
  !gapprompt@gap>| !gapinput@H := QuaternionAlgebra(Rationals); AsList(Basis(H));|
  <algebra-with-one of dimension 4 over Rationals>
  [ e, i, j, k ]
  !gapprompt@gap>| !gapinput@List(Basis(H), x -> ComplexConjugate(x));|
  [ e, (-1)*i, (-1)*j, (-1)*k ]
  !gapprompt@gap>| !gapinput@List(Basis(H), x -> RealPart(x));|
  [ e, 0*e, 0*e, 0*e ]
  !gapprompt@gap>| !gapinput@List(Basis(H), x -> ImaginaryPart(x));|
  [ 0*e, i, j, k ]
\end{Verbatim}
 }

 

\subsection{\textcolor{Chapter }{QuaternionD4basis}}
\logpage{[ 2, 4, 6 ]}\nobreak
\hyperdef{L}{X78FF8724803E2AB4}{}
{\noindent\textcolor{FuncColor}{$\triangleright$\enspace\texttt{QuaternionD4basis\index{QuaternionD4basis@\texttt{QuaternionD4basis}}
\label{QuaternionD4basis}
}\hfill{\scriptsize (global variable)}}\\


 The \textsf{ALCO} package loads a basis for a quaternion algebra over {\ensuremath{\mathbb Q}}
with the geometry of a $D_4$ simpe root system. The {\ensuremath{\mathbb Z}}-span of this basis is the
Hurwitz ring. These basis vectors close under pairwise reflection to form a $ D_4$ root system. 
\begin{Verbatim}[commandchars=!@|,fontsize=\small,frame=single,label=Example]
  !gapprompt@gap>| !gapinput@B := QuaternionD4basis;;|
  !gapprompt@gap>| !gapinput@for x in BasisVectors(B) do Display(x); od;|
  (-1/2)*e+(-1/2)*i+(-1/2)*j+(1/2)*k
  (-1/2)*e+(-1/2)*i+(1/2)*j+(-1/2)*k
  (-1/2)*e+(1/2)*i+(-1/2)*j+(-1/2)*k
  e
\end{Verbatim}
 }

 
\subsection{\textcolor{Chapter }{Golden Field Values}}\logpage{[ 2, 4, 7 ]}
\hyperdef{L}{X80F5CF497843D181}{}
{
\noindent\textcolor{FuncColor}{$\triangleright$\enspace\texttt{sigma\index{sigma@\texttt{sigma}}
\label{sigma}
}\hfill{\scriptsize (global variable)}}\\
\noindent\textcolor{FuncColor}{$\triangleright$\enspace\texttt{tau\index{tau@\texttt{tau}}
\label{tau}
}\hfill{\scriptsize (global variable)}}\\


 The \textsf{ALCO} package loads the following elements of the golden field, \texttt{\symbol{92}}\texttt{NF(5,[ 1, 4 ])}: 
\begin{Verbatim}[commandchars=!@|,fontsize=\small,frame=single,label=Example]
  !gapprompt@gap>| !gapinput@TeachingMode(true);|
  #I  Teaching mode is turned ON
  !gapprompt@gap>| !gapinput@sigma;|
  (1-Sqrt(5))/2
  !gapprompt@gap>| !gapinput@tau;|
  (1+Sqrt(5))/2 
  !gapprompt@gap>| !gapinput@Field(sigma);|
  NF(5,[ 1, 4 ])
  !gapprompt@gap>| !gapinput@Field(tau);|
  NF(5,[ 1, 4 ])
  !gapprompt@gap>| !gapinput@Field(Sqrt(5));|
  NF(5,[ 1, 4 ])
\end{Verbatim}
 }

 
\subsection{\textcolor{Chapter }{GoldenModSigma}}\logpage{[ 2, 4, 8 ]}
\hyperdef{L}{X7C5123127E6FFFA7}{}
{
\noindent\textcolor{FuncColor}{$\triangleright$\enspace\texttt{GoldenModSigma({\mdseries\slshape x})\index{GoldenModSigma@\texttt{GoldenModSigma}}
\label{GoldenModSigma}
}\hfill{\scriptsize (function)}}\\


 For \mbox{\texttt{\mdseries\slshape x}} in the golden field \texttt{NF(5,[ 1, 4 ])}, this function returns the rational coefficient of \texttt{1} in the basis \texttt{Basis(NF(5,[ 1, 4 ]), [1, sigma])}. 
\begin{Verbatim}[commandchars=!@|,fontsize=\small,frame=single,label=Example]
  !gapprompt@gap>| !gapinput@x := 5 + 3*sigma;; GoldenModSigma(x);|
  5 
  !gapprompt@gap>| !gapinput@GoldenModSigma(sigma);|
  0
  !gapprompt@gap>| !gapinput@GoldenModSigma(tau);|
  1
\end{Verbatim}
 }

 

\subsection{\textcolor{Chapter }{IcosianH4basis}}
\logpage{[ 2, 4, 9 ]}\nobreak
\hyperdef{L}{X7F28DE91825770B6}{}
{\noindent\textcolor{FuncColor}{$\triangleright$\enspace\texttt{IcosianH4basis\index{IcosianH4basis@\texttt{IcosianH4basis}}
\label{IcosianH4basis}
}\hfill{\scriptsize (global variable)}}\\


 The \textsf{ALCO} package loads a basis for a quaternion algebra over \texttt{NF(5,[1,4])}. The {\ensuremath{\mathbb Z}}-span of this basis is the icosian ring. These
basis vectors close under pairwise reflection to form a $H_4$ set of vectors. 
\begin{Verbatim}[commandchars=!@|,fontsize=\small,frame=single,label=Example]
  !gapprompt@gap>| !gapinput@B := IcosianH4basis;;|
  !gapprompt@gap>| !gapinput@for x in BasisVectors(B) do Display(x); od;|
  (-1)*i
  (-1/2*E(5)^2-1/2*E(5)^3)*i+(1/2)*j+(-1/2*E(5)-1/2*E(5)^4)*k
  (-1)*j
  (-1/2*E(5)-1/2*E(5)^4)*e+(1/2)*j+(-1/2*E(5)^2-1/2*E(5)^3)*k
\end{Verbatim}
 }

 }

 }

 
\chapter{\textcolor{Chapter }{Simple Euclidean Jordan Algebras}}\logpage{[ 3, 0, 0 ]}
\hyperdef{L}{X7E13C2AE7DEAF62D}{}
{
  Simple Euclidean Jordan algebras are described well in \cite{faraut_analysis_1994}. 
\section{\textcolor{Chapter }{Filters and Basic Attributes}}\logpage{[ 3, 1, 0 ]}
\hyperdef{L}{X802D4E3380BC3177}{}
{
  
\subsection{\textcolor{Chapter }{Jordan Filters}}\logpage{[ 3, 1, 1 ]}
\hyperdef{L}{X878107A77FBFD00A}{}
{
\noindent\textcolor{FuncColor}{$\triangleright$\enspace\texttt{IsJordanAlgebra\index{IsJordanAlgebra@\texttt{IsJordanAlgebra}}
\label{IsJordanAlgebra}
}\hfill{\scriptsize (filter)}}\\
\noindent\textcolor{FuncColor}{$\triangleright$\enspace\texttt{IsJordanAlgebraObj\index{IsJordanAlgebraObj@\texttt{IsJordanAlgebraObj}}
\label{IsJordanAlgebraObj}
}\hfill{\scriptsize (filter)}}\\


These filters determine whether an element is a Jordan algebra (\texttt{IsJordanAlgebra}) or is an element in a Jordan algebra (\texttt{IsJordanAlgebraObj}). }

 A simple Euclidean Jordan algebra $V$ has rank $r$ and degree $d$. The following methods return the properties of either a Jordan algebra or of
the Jordan algebra containing the object. 

\subsection{\textcolor{Chapter }{JordanRank}}
\logpage{[ 3, 1, 2 ]}\nobreak
\hyperdef{L}{X78D142B07F59FFAF}{}
{\noindent\textcolor{FuncColor}{$\triangleright$\enspace\texttt{JordanRank({\mdseries\slshape x})\index{JordanRank@\texttt{JordanRank}}
\label{JordanRank}
}\hfill{\scriptsize (method)}}\\


Returns the rank of \mbox{\texttt{\mdseries\slshape x}} when \texttt{IsJordanAlgebra(x)} or the rank of the Jordan algebra containing \mbox{\texttt{\mdseries\slshape x}} when \texttt{IsJordanAlgebraObj(x)}. }

 

\subsection{\textcolor{Chapter }{JordanDegree}}
\logpage{[ 3, 1, 3 ]}\nobreak
\hyperdef{L}{X872B09017D7C2E5C}{}
{\noindent\textcolor{FuncColor}{$\triangleright$\enspace\texttt{JordanDegree({\mdseries\slshape x})\index{JordanDegree@\texttt{JordanDegree}}
\label{JordanDegree}
}\hfill{\scriptsize (method)}}\\


Returns the degree of \mbox{\texttt{\mdseries\slshape x}} when \texttt{IsJordanAlgebra(x)} or the degree of the Jordan algebra containing \mbox{\texttt{\mdseries\slshape x}} when \texttt{IsJordanAlgebraObj(x)} . }

 

\subsection{\textcolor{Chapter }{Trace (Jordan Algebras)}}
\logpage{[ 3, 1, 4 ]}\nobreak
\hyperdef{L}{X80051D4E7B64E102}{}
{\noindent\textcolor{FuncColor}{$\triangleright$\enspace\texttt{Trace({\mdseries\slshape x})\index{Trace@\texttt{Trace}!Jordan Algebras}
\label{Trace:Jordan Algebras}
}\hfill{\scriptsize (method)}}\\


Returns the Jordan trace of \mbox{\texttt{\mdseries\slshape x}} when \texttt{IsJordanAlgebraObj(x)}.}

 

\subsection{\textcolor{Chapter }{Norm (Jordan Algebras)}}
\logpage{[ 3, 1, 5 ]}\nobreak
\hyperdef{L}{X83B5D76B87AEF802}{}
{\noindent\textcolor{FuncColor}{$\triangleright$\enspace\texttt{Norm({\mdseries\slshape x})\index{Norm@\texttt{Norm}!Jordan Algebras}
\label{Norm:Jordan Algebras}
}\hfill{\scriptsize (method)}}\\


Returns the Jordan norm of \mbox{\texttt{\mdseries\slshape x}} when \texttt{IsJordanAlgebraObj(x)}. The Jordan norm has the value \texttt{Trace(x\texttt{\symbol{94}}2)/2}. }

 

\subsection{\textcolor{Chapter }{GenericMinimalPolynomial}}
\logpage{[ 3, 1, 6 ]}\nobreak
\hyperdef{L}{X85D508B5853906E5}{}
{\noindent\textcolor{FuncColor}{$\triangleright$\enspace\texttt{GenericMinimalPolynomial({\mdseries\slshape x})\index{GenericMinimalPolynomial@\texttt{GenericMinimalPolynomial}}
\label{GenericMinimalPolynomial}
}\hfill{\scriptsize (attribute)}}\\


 Returns the generic minimal polynomial of \mbox{\texttt{\mdseries\slshape x}} when \texttt{IsJordanAlgebraObj(x)} as defined in \cite[p. 478]{faraut_analysis_2000}. The output is given as a list of polynomial coefficients. }

 

\subsection{\textcolor{Chapter }{Determinant (Jordan Algebras)}}
\logpage{[ 3, 1, 7 ]}\nobreak
\hyperdef{L}{X844D03667EC7C372}{}
{\noindent\textcolor{FuncColor}{$\triangleright$\enspace\texttt{Determinant({\mdseries\slshape x})\index{Determinant@\texttt{Determinant}!Jordan Algebras}
\label{Determinant:Jordan Algebras}
}\hfill{\scriptsize (method)}}\\


Returns the Jordan determinant of \mbox{\texttt{\mdseries\slshape x}} when \texttt{IsJordanAlgebraObj(x)} .}

 }

 
\section{\textcolor{Chapter }{Jordan Algebra Constructions}}\logpage{[ 3, 2, 0 ]}
\hyperdef{L}{X7FBD095A7B884F7F}{}
{
  

\subsection{\textcolor{Chapter }{SimpleEuclideanJordanAlgebra}}
\logpage{[ 3, 2, 1 ]}\nobreak
\hyperdef{L}{X7852050A81DEB9F4}{}
{\noindent\textcolor{FuncColor}{$\triangleright$\enspace\texttt{SimpleEuclideanJordanAlgebra({\mdseries\slshape rho, d[, args]})\index{SimpleEuclideanJordanAlgebra@\texttt{SimpleEuclideanJordanAlgebra}}
\label{SimpleEuclideanJordanAlgebra}
}\hfill{\scriptsize (function)}}\\


 Returns a simple Euclidean Jordan algebra over {\ensuremath{\mathbb Q}} in an
orthogonal basis. 
\begin{Verbatim}[commandchars=!@|,fontsize=\small,frame=single,label=Example]
  !gapprompt@gap>| !gapinput@J := SimpleEuclideanJordanAlgebra(3,8);|
  <algebra of dimension 27 over Rationals>
  !gapprompt@gap>| !gapinput@SemiSimpleType(Derivations(Basis(J)));|
  "F4"
\end{Verbatim}
 }

 

\subsection{\textcolor{Chapter }{JordanSpinFactor}}
\logpage{[ 3, 2, 2 ]}\nobreak
\hyperdef{L}{X86C6713C8178A69F}{}
{\noindent\textcolor{FuncColor}{$\triangleright$\enspace\texttt{JordanSpinFactor({\mdseries\slshape G})\index{JordanSpinFactor@\texttt{JordanSpinFactor}}
\label{JordanSpinFactor}
}\hfill{\scriptsize (function)}}\\


 Returns a Jordan spin factor algebra when \mbox{\texttt{\mdseries\slshape G}} is a positive definite Gram matrix. 
\begin{Verbatim}[commandchars=!@|,fontsize=\small,frame=single,label=Example]
  !gapprompt@gap>| !gapinput@J := JordanSpinFactor(IdentityMat(8));|
  <algebra of dimension 9 over Rationals>
  !gapprompt@gap>| !gapinput@One(J);|
  v.1
  !gapprompt@gap>| !gapinput@[JordanRank(J), JordanDegree(J)];|
  [ 2, 7 ]
  !gapprompt@gap>| !gapinput@Derivations(Basis(J));|
  <Lie algebra of dimension 28 over Rationals>
  !gapprompt@gap>| !gapinput@SemiSimpleType(last);|
  "D4"
  !gapprompt@gap>| !gapinput@x := Random(J);|
  v.2+(-1)*v.3+(-1)*v.4+(1/2)*v.5+(-2)*v.7+(1/2)*v.8+(-3/2)*v.9
  !gapprompt@gap>| !gapinput@[Trace(x), Determinant(x)];|
  [ 0, -39/4 ]
  !gapprompt@gap>| !gapinput@p := GenericMinimalPolynomial(x);|
  [ -39/4, 0, 1 ]
  !gapprompt@gap>| !gapinput@ValuePol(p, x);|
  0*v.1
\end{Verbatim}
 }

 

\subsection{\textcolor{Chapter }{HermitianSimpleJordanAlgebra}}
\logpage{[ 3, 2, 3 ]}\nobreak
\hyperdef{L}{X859F001D7CB6CBD8}{}
{\noindent\textcolor{FuncColor}{$\triangleright$\enspace\texttt{HermitianSimpleJordanAlgebra({\mdseries\slshape r, B})\index{HermitianSimpleJordanAlgebra@\texttt{HermitianSimpleJordanAlgebra}}
\label{HermitianSimpleJordanAlgebra}
}\hfill{\scriptsize (function)}}\\


 Returns a simple Euclidean Jordan algebra of rank \mbox{\texttt{\mdseries\slshape r}} with the basis for the off-diagonal components defined using composition
algebra basis \mbox{\texttt{\mdseries\slshape B}}. 
\begin{Verbatim}[commandchars=!@|,fontsize=\small,frame=single,label=Example]
  !gapprompt@gap>| !gapinput@B := OctonionE8basis;;|
  !gapprompt@gap>| !gapinput@J := HermitianSimpleJordanAlgebra(3,B);|
  <algebra of dimension 27 over Rationals>
  !gapprompt@gap>| !gapinput@[JordanRank(J), JordanDegree(J)];|
  [ 3, 8 ]
  !gapprompt@gap>| !gapinput@Derivations(Basis(J));|
  <Lie algebra of dimension 52 over Rationals>
  !gapprompt@gap>| !gapinput@SemiSimpleType(last);|
  "F4"
\end{Verbatim}
 }

 

\subsection{\textcolor{Chapter }{JordanHomotope}}
\logpage{[ 3, 2, 4 ]}\nobreak
\hyperdef{L}{X800B48C383196E06}{}
{\noindent\textcolor{FuncColor}{$\triangleright$\enspace\texttt{JordanHomotope({\mdseries\slshape J, u[, s]})\index{JordanHomotope@\texttt{JordanHomotope}}
\label{JordanHomotope}
}\hfill{\scriptsize (function)}}\\


 For \mbox{\texttt{\mdseries\slshape J}} a Jordan algebra satisfying \texttt{IsJordanAlgebra(\mbox{\texttt{\mdseries\slshape J}})}, and for \mbox{\texttt{\mdseries\slshape u}} a vector in \mbox{\texttt{\mdseries\slshape J}}, this function returns the corresponding \mbox{\texttt{\mdseries\slshape u}}-homotope algebra with the product of $x$ and $y$ defined as $x(uy)+(xu)y - u(xy)$. The \mbox{\texttt{\mdseries\slshape u}}-homotope algebra also belongs to the filter \texttt{IsJordanAlgebra}. Of note, if \mbox{\texttt{\mdseries\slshape u}} is invertible in \mbox{\texttt{\mdseries\slshape J}} then the corresponding \mbox{\texttt{\mdseries\slshape u}}-homotope algebra is called a \mbox{\texttt{\mdseries\slshape u}}-isotope. The optional argument \mbox{\texttt{\mdseries\slshape s}} is a string that determines the labels of the canonical basis vectors in the
new algebra. 
\begin{Verbatim}[commandchars=!@|,fontsize=\small,frame=single,label=Example]
  !gapprompt@gap>| !gapinput@J := SimpleEuclideanJordanAlgebra(2,7);|
  <algebra of dimension 9 over Rationals>
  !gapprompt@gap>| !gapinput@u := Random(J);|
  (-1/6)*v.1+(3)*v.2+(1/3)*v.3+(-2)*v.4+(-4)*v.6+(-1)*v.8+(-3)*v.9
  !gapprompt@gap>| !gapinput@GenericMinimalPolynomial(u);|
  [ -469/12, 1/3, 1 ]
  !gapprompt@gap>| !gapinput@H := JordanHomotope(J, u);|
  <algebra of dimension 9 over Rationals>
  !gapprompt@gap>| !gapinput@SemiSimpleType(Derivations(Basis(J)));|
  "D4"
  !gapprompt@gap>| !gapinput@SemiSimpleType(Derivations(Basis(H)));|
  "D4"
\end{Verbatim}
 }

 

\subsection{\textcolor{Chapter }{AlbertAlgebra}}
\logpage{[ 3, 2, 5 ]}\nobreak
\hyperdef{L}{X7A6AFFE07994B4A9}{}
{\noindent\textcolor{FuncColor}{$\triangleright$\enspace\texttt{AlbertAlgebra({\mdseries\slshape F})\index{AlbertAlgebra@\texttt{AlbertAlgebra}}
\label{AlbertAlgebra}
}\hfill{\scriptsize (function)}}\\


 For \mbox{\texttt{\mdseries\slshape F}} a field, this function returns an Albert algebra over \mbox{\texttt{\mdseries\slshape F}}. This algebra is isomorphic to \texttt{HermitianSimpleJordanAlgebra(3,8,Basis(Oct))} but in a basis that is more convenient for reproducing certain calculations in
the literature. }

 

\subsection{\textcolor{Chapter }{Alb}}
\logpage{[ 3, 2, 6 ]}\nobreak
\hyperdef{L}{X7B456B2582F195DB}{}
{\noindent\textcolor{FuncColor}{$\triangleright$\enspace\texttt{Alb\index{Alb@\texttt{Alb}}
\label{Alb}
}\hfill{\scriptsize (global variable)}}\\


 The \textsf{ALCO} package includes a loaded instance of the Albert algebra over the rationals. 
\begin{Verbatim}[commandchars=!@|,fontsize=\small,frame=single,label=Example]
  !gapprompt@gap>| !gapinput@Alb;|
  <algebra of dimension 27 over Rationals>
\end{Verbatim}
 }

 }

 
\section{\textcolor{Chapter }{Additional Tools and Properties}}\logpage{[ 3, 3, 0 ]}
\hyperdef{L}{X7EA1D48F853C02F1}{}
{
  

\subsection{\textcolor{Chapter }{HermitianJordanAlgebraBasis}}
\logpage{[ 3, 3, 1 ]}\nobreak
\hyperdef{L}{X7C236EB080D05CD4}{}
{\noindent\textcolor{FuncColor}{$\triangleright$\enspace\texttt{HermitianJordanAlgebraBasis({\mdseries\slshape r, B})\index{HermitianJordanAlgebraBasis@\texttt{HermitianJordanAlgebraBasis}}
\label{HermitianJordanAlgebraBasis}
}\hfill{\scriptsize (function)}}\\


 Returns a set of Hermitian matrices to serve as a basis for the Jordan algebra
with or rank \mbox{\texttt{\mdseries\slshape r}} and degree given by the cardinality of composition algebra basis \mbox{\texttt{\mdseries\slshape B}}. The elements spanning each off-diagonal components are determined by basis \mbox{\texttt{\mdseries\slshape B}}. 
\begin{Verbatim}[commandchars=!@|,fontsize=\small,frame=single,label=Example]
  !gapprompt@gap>| !gapinput@H := QuaternionAlgebra(Rationals);; AsList(Basis(H));|
  [ e, i, j, k ]
  !gapprompt@gap>| !gapinput@for x in HermitianJordanAlgebraBasis(2, Basis(H)) do Display(x); od;|
  [ [    e,  0*e ],
    [  0*e,  0*e ] ]
  [ [  0*e,  0*e ],
    [  0*e,    e ] ]
  [ [  0*e,    e ],
    [    e,  0*e ] ]
  [ [     0*e,       i ],
    [  (-1)*i,     0*e ] ]
  [ [     0*e,       j ],
    [  (-1)*j,     0*e ] ]
  [ [     0*e,       k ],
    [  (-1)*k,     0*e ] ]
\end{Verbatim}
 }

 

\subsection{\textcolor{Chapter }{JordanMatrixBasis}}
\logpage{[ 3, 3, 2 ]}\nobreak
\hyperdef{L}{X853480DC7F9B0BD7}{}
{\noindent\textcolor{FuncColor}{$\triangleright$\enspace\texttt{JordanMatrixBasis({\mdseries\slshape J})\index{JordanMatrixBasis@\texttt{JordanMatrixBasis}}
\label{JordanMatrixBasis}
}\hfill{\scriptsize (attribute)}}\\


 If \texttt{IsJordanAlgebra( \mbox{\texttt{\mdseries\slshape J}} )} and \mbox{\texttt{\mdseries\slshape J}} has been constructed using a matrix basis, then the set of matrices
corresponding to \texttt{CanonicalBasis( \mbox{\texttt{\mdseries\slshape J}} )} can be obtained using \texttt{JordanMatrixBasis( \mbox{\texttt{\mdseries\slshape J}} )}. }

 

\subsection{\textcolor{Chapter }{HermitianMatrixToJordanCoefficients}}
\logpage{[ 3, 3, 3 ]}\nobreak
\hyperdef{L}{X8031519B84E53A57}{}
{\noindent\textcolor{FuncColor}{$\triangleright$\enspace\texttt{HermitianMatrixToJordanCoefficients({\mdseries\slshape mat, J})\index{HermitianMatrixToJordanCoefficients@\texttt{HermitianMatrixToJordanCoefficients}}
\label{HermitianMatrixToJordanCoefficients}
}\hfill{\scriptsize (function)}}\\


 Converts matrix \mbox{\texttt{\mdseries\slshape mat}} into an element of Jordan algebra \mbox{\texttt{\mdseries\slshape J}}. 
\begin{Verbatim}[commandchars=!@|,fontsize=\small,frame=single,label=Example]
               
  !gapprompt@gap>| !gapinput@H := QuaternionAlgebra(Rationals);; AsList(Basis(H));|
  [ e, i, j, k ]
  !gapprompt@gap>| !gapinput@J := HermitianSimpleJordanAlgebra(2,Basis(H));|
  <algebra of dimension 6 over Rationals>
  !gapprompt@gap>| !gapinput@AsList(CanonicalBasis(J));|
  [ v.1, v.2, v.3, v.4, v.5, v.6 ]
  !gapprompt@gap>| !gapinput@JordanMatrixBasis(J);|
  [ [ [ e, 0*e ], [ 0*e, 0*e ] ],    [ [ 0*e, 0*e ], [ 0*e, e ] ], 
    [ [ 0*e, e ], [ e, 0*e ] ],      [ [ 0*e, i ], [ (-1)*i, 0*e ] ], 
    [ [ 0*e, j ], [ (-1)*j, 0*e ] ], [ [ 0*e, k ], [ (-1)*k, 0*e ] ] ]
  !gapprompt@gap>| !gapinput@List(JordanMatrixBasis(J), x -> HermitianMatrixToJordanCoefficients(x, J));|
  [ v.1, v.2, v.3, v.4, v.5, v.6 ]
\end{Verbatim}
 }

 

\subsection{\textcolor{Chapter }{JordanAlgebraGramMatrix}}
\logpage{[ 3, 3, 4 ]}\nobreak
\hyperdef{L}{X7B6084887C1C5AF1}{}
{\noindent\textcolor{FuncColor}{$\triangleright$\enspace\texttt{JordanAlgebraGramMatrix({\mdseries\slshape J})\index{JordanAlgebraGramMatrix@\texttt{JordanAlgebraGramMatrix}}
\label{JordanAlgebraGramMatrix}
}\hfill{\scriptsize (attribute)}}\\


 For \texttt{IsJordanAlgebra( \mbox{\texttt{\mdseries\slshape J}} )}, returns the Gram matrix on \texttt{CanonicalBasis( \mbox{\texttt{\mdseries\slshape J}} )} using inner product \texttt{Trace(x*y)}. 
\begin{Verbatim}[commandchars=!@|,fontsize=\small,frame=single,label=Example]
  !gapprompt@gap>| !gapinput@J := HermitianSimpleJordanAlgebra(2,OctonionE8basis);|
  <algebra of dimension 10 over Rationals>
  !gapprompt@gap>| !gapinput@Display(JordanAlgebraGramMatrix(J));|
  [ [   1,   0,   0,   0,   0,   0,   0,   0,   0,   0 ],
    [   0,   1,   0,   0,   0,   0,   0,   0,   0,   0 ],
    [   0,   0,   2,   0,  -1,   0,   0,   0,   0,   0 ],
    [   0,   0,   0,   2,   0,  -1,   0,   0,   0,   0 ],
    [   0,   0,  -1,   0,   2,  -1,   0,   0,   0,   0 ],
    [   0,   0,   0,  -1,  -1,   2,  -1,   0,   0,   0 ],
    [   0,   0,   0,   0,   0,  -1,   2,  -1,   0,   0 ],
    [   0,   0,   0,   0,   0,   0,  -1,   2,  -1,   0 ],
    [   0,   0,   0,   0,   0,   0,   0,  -1,   2,  -1 ],
    [   0,   0,   0,   0,   0,   0,   0,   0,  -1,   2 ] ]  
\end{Verbatim}
 }

 

\subsection{\textcolor{Chapter }{JordanAdjugate}}
\logpage{[ 3, 3, 5 ]}\nobreak
\hyperdef{L}{X806F76C07B315DE4}{}
{\noindent\textcolor{FuncColor}{$\triangleright$\enspace\texttt{JordanAdjugate({\mdseries\slshape x})\index{JordanAdjugate@\texttt{JordanAdjugate}}
\label{JordanAdjugate}
}\hfill{\scriptsize (function)}}\\


 For \texttt{IsJordanAlgebraObj( \mbox{\texttt{\mdseries\slshape x}} )}, returns the adjugate of \mbox{\texttt{\mdseries\slshape x}}, which satisfies \texttt{x*JordanAdjugate(x) = One(x)*Determinant(x)}. When \texttt{Determinant(x)} is non-zero, \texttt{JordanAdjugate(x)} is proportional to \texttt{Inverse(x)}. }

 

\subsection{\textcolor{Chapter }{IsPositiveDefinite}}
\logpage{[ 3, 3, 6 ]}\nobreak
\hyperdef{L}{X7C82B2EB78CF9C17}{}
{\noindent\textcolor{FuncColor}{$\triangleright$\enspace\texttt{IsPositiveDefinite({\mdseries\slshape x})\index{IsPositiveDefinite@\texttt{IsPositiveDefinite}}
\label{IsPositiveDefinite}
}\hfill{\scriptsize (filter)}}\\


 For \texttt{IsJordanAlgebraObj( \mbox{\texttt{\mdseries\slshape x}} )}, returns \texttt{true} when \mbox{\texttt{\mdseries\slshape x}} is positive definite and \texttt{false} otherwise. This filter uses \texttt{GenericMinimalPolynomial} (\ref{GenericMinimalPolynomial}) to determine whether \mbox{\texttt{\mdseries\slshape x}} is positive definite. }

 }

 }

 
\chapter{\textcolor{Chapter }{Jordan Designs and their Association Schemes}}\logpage{[ 4, 0, 0 ]}
\hyperdef{L}{X7A69764E7EEFC0E0}{}
{
  
\section{\textcolor{Chapter }{Jacobi Polynomials}}\logpage{[ 4, 1, 0 ]}
\hyperdef{L}{X79A71E957D5B9755}{}
{
  

\subsection{\textcolor{Chapter }{JacobiPolynomial}}
\logpage{[ 4, 1, 1 ]}\nobreak
\hyperdef{L}{X872DC5F085155040}{}
{\noindent\textcolor{FuncColor}{$\triangleright$\enspace\texttt{JacobiPolynomial({\mdseries\slshape k, a, b})\index{JacobiPolynomial@\texttt{JacobiPolynomial}}
\label{JacobiPolynomial}
}\hfill{\scriptsize (function)}}\\


 This function returns the Jacobi polynomial $P(x)$ of degree \mbox{\texttt{\mdseries\slshape k}} and type \mbox{\texttt{\mdseries\slshape (a,b)}} as defined in \cite[chap. 22]{abramowitz_handbook_1972}. 
\begin{Verbatim}[commandchars=!@|,fontsize=\small,frame=single,label=Example]
  !gapprompt@gap>| !gapinput@a := Indeterminate(Rationals, "a");; |
  !gapprompt@gap>| !gapinput@b := Indeterminate(Rationals, "b");; |
  !gapprompt@gap>| !gapinput@x := Indeterminate(Rationals, "x");;|
  !gapprompt@gap>| !gapinput@JacobiPolynomial(0,a,b);|
  [ 1 ]
  !gapprompt@gap>| !gapinput@JacobiPolynomial(1,a,b);|
  [ 1/2*a-1/2*b, 1/2*a+1/2*b+1 ]
  !gapprompt@gap>| !gapinput@ValuePol(last,x);|
  1/2*a*x+1/2*b*x+1/2*a-1/2*b+x 
\end{Verbatim}
 }

 
\subsection{\textcolor{Chapter }{Renormalized Jacobi Polynomials}}\logpage{[ 4, 1, 2 ]}
\hyperdef{L}{X7C78C3A57DDC372B}{}
{
\noindent\textcolor{FuncColor}{$\triangleright$\enspace\texttt{Q{\textunderscore}k{\textunderscore}epsilon({\mdseries\slshape k, epsilon, rank, degree, x})\index{Q{\textunderscore}k{\textunderscore}epsilon@\texttt{Q{\textunderscore}k{\textunderscore}epsilon}}
\label{QuScorekuScoreepsilon}
}\hfill{\scriptsize (function)}}\\
\noindent\textcolor{FuncColor}{$\triangleright$\enspace\texttt{R{\textunderscore}k{\textunderscore}epsilon({\mdseries\slshape k, epsilon, rank, degree, x})\index{R{\textunderscore}k{\textunderscore}epsilon@\texttt{R{\textunderscore}k{\textunderscore}epsilon}}
\label{RuScorekuScoreepsilon}
}\hfill{\scriptsize (function)}}\\


 These functions return polynomials of degree \mbox{\texttt{\mdseries\slshape k}} in the indeterminate \mbox{\texttt{\mdseries\slshape x}} corresponding the the renormalized Jacobi polynomials given in \cite{hoggar_t-designs_1982}. The value of \mbox{\texttt{\mdseries\slshape epsilon}} must be 0 or 1. The arguments \mbox{\texttt{\mdseries\slshape rank}} and \mbox{\texttt{\mdseries\slshape degree}} correspond to the rank and degree of the relevant simple Euclidean Jordan
algebra. }

 }

 
\section{\textcolor{Chapter }{Jordan Designs}}\logpage{[ 4, 2, 0 ]}
\hyperdef{L}{X83D7DAD082EAD26D}{}
{
  
\subsection{\textcolor{Chapter }{Jordan Design Filters}}\logpage{[ 4, 2, 1 ]}
\hyperdef{L}{X79A341E17ECCAC34}{}
{
\noindent\textcolor{FuncColor}{$\triangleright$\enspace\texttt{IsDesign\index{IsDesign@\texttt{IsDesign}}
\label{IsDesign}
}\hfill{\scriptsize (filter)}}\\
\noindent\textcolor{FuncColor}{$\triangleright$\enspace\texttt{IsSphericalDesign\index{IsSphericalDesign@\texttt{IsSphericalDesign}}
\label{IsSphericalDesign}
}\hfill{\scriptsize (filter)}}\\
\noindent\textcolor{FuncColor}{$\triangleright$\enspace\texttt{IsProjectiveDesign\index{IsProjectiveDesign@\texttt{IsProjectiveDesign}}
\label{IsProjectiveDesign}
}\hfill{\scriptsize (filter)}}\\


These filters determine whether an object is a Jordan design and whether the
design is constructed in a spherical or projective manifold of Jordan
primitive idempotents.}

 

\subsection{\textcolor{Chapter }{DesignByJordanParameters}}
\logpage{[ 4, 2, 2 ]}\nobreak
\hyperdef{L}{X7F44B7A981CECE35}{}
{\noindent\textcolor{FuncColor}{$\triangleright$\enspace\texttt{DesignByJordanParameters({\mdseries\slshape rank, degree})\index{DesignByJordanParameters@\texttt{DesignByJordanParameters}}
\label{DesignByJordanParameters}
}\hfill{\scriptsize (function)}}\\


 This function constructs a Jordan design in the manifold of Jordan primitive
idempotents of rank \mbox{\texttt{\mdseries\slshape rank}} and degree \mbox{\texttt{\mdseries\slshape degree}}. 
\begin{Verbatim}[commandchars=!@|,fontsize=\small,frame=single,label=Example]
  !gapprompt@gap>| !gapinput@D := DesignByJordanParameters(3,8);|
  <design with rank 3 and degree 8>
  !gapprompt@gap>| !gapinput@IsDesign(D);|
  true
  !gapprompt@gap>| !gapinput@IsSphericalDesign(D);|
  false
  !gapprompt@gap>| !gapinput@IsProjectiveDesign(D);|
  true 
\end{Verbatim}
 }

 
\subsection{\textcolor{Chapter }{Jordan Rank and Degree}}\logpage{[ 4, 2, 3 ]}
\hyperdef{L}{X82BB5F997B7B7B3A}{}
{
\noindent\textcolor{FuncColor}{$\triangleright$\enspace\texttt{DesignJordanRank({\mdseries\slshape D})\index{DesignJordanRank@\texttt{DesignJordanRank}}
\label{DesignJordanRank}
}\hfill{\scriptsize (attribute)}}\\
\noindent\textcolor{FuncColor}{$\triangleright$\enspace\texttt{DesignJordanDegree({\mdseries\slshape D})\index{DesignJordanDegree@\texttt{DesignJordanDegree}}
\label{DesignJordanDegree}
}\hfill{\scriptsize (attribute)}}\\


 The rank and degree of an object satisfying filter \texttt{IsDesign} are stored as attributes. 
\begin{Verbatim}[commandchars=!@|,fontsize=\small,frame=single,label=Example]
  !gapprompt@gap>| !gapinput@D := DesignByJordanParameters(3,8);|
  <design with rank 3 and degree 8>
  !gapprompt@gap>| !gapinput@[DesignJordanRank(D), DesignJordanDegree(D)];|
  [ 3, 8 ] 
\end{Verbatim}
 }

 

\subsection{\textcolor{Chapter }{DesignQPolynomial}}
\logpage{[ 4, 2, 4 ]}\nobreak
\hyperdef{L}{X83768DF683E353F7}{}
{\noindent\textcolor{FuncColor}{$\triangleright$\enspace\texttt{DesignQPolynomial({\mdseries\slshape D})\index{DesignQPolynomial@\texttt{DesignQPolynomial}}
\label{DesignQPolynomial}
}\hfill{\scriptsize (attribute)}}\\


 This attribute stores a function on non-negative integers that returns the
coefficients of the renormalized Jacobi polynomial in the manifold of Jordan
primitive idempotents corresponding to the design \mbox{\texttt{\mdseries\slshape D}}. 
\begin{Verbatim}[commandchars=!@|,fontsize=\small,frame=single,label=Example]
  !gapprompt@gap>| !gapinput@D := DesignByJordanParameters(3,8);|
  <design with rank 3 and degree 8>
  !gapprompt@gap>| !gapinput@r := DesignJordanRank(D);; d := DesignJordanDegree(D);; |
  !gapprompt@gap>| !gapinput@x := Indeterminate(Rationals, "x");;|
  !gapprompt@gap>| !gapinput@DesignQPolynomial(D);|
  function( k ) ... end
  !gapprompt@gap>| !gapinput@DesignQPolynomial(D)(2);|
  [ 90, -585, 819 ]
  !gapprompt@gap>| !gapinput@CoefficientsOfUnivariatePolynomial(Q_k_epsilon(2,0,r,d,x));|
  [ 90, -585, 819 ]
\end{Verbatim}
 }

 

\subsection{\textcolor{Chapter }{DesignConnectionCoefficients}}
\logpage{[ 4, 2, 5 ]}\nobreak
\hyperdef{L}{X7E14BB6A80B452AF}{}
{\noindent\textcolor{FuncColor}{$\triangleright$\enspace\texttt{DesignConnectionCoefficients({\mdseries\slshape D})\index{DesignConnectionCoefficients@\texttt{DesignConnectionCoefficients}}
\label{DesignConnectionCoefficients}
}\hfill{\scriptsize (attribute)}}\\


 This attribute stores the connection coefficients, defined in \cite[p. 261]{hoggar_t-designs_1992}, which determine the linear combinations of \texttt{DesignQPolynomial(\mbox{\texttt{\mdseries\slshape D}})} polynomials that yield each power of the indeterminate. 
\begin{Verbatim}[commandchars=!@|,fontsize=\small,frame=single,label=Example]
  !gapprompt@gap>| !gapinput@D := DesignByJordanParameters(3,8);|
  <design with rank 3 and degree 8>
  !gapprompt@gap>| !gapinput@DesignConnectionCoefficients(D);|
  function( s ) ... end
  !gapprompt@gap>| !gapinput@f := DesignConnectionCoefficients(D)(3);|
  [ [ 1, 0, 0, 0 ], [ 1/3, 1/39, 0, 0 ], [ 5/39, 5/273, 1/819, 0 ],
    [ 5/91, 1/91, 1/728, 1/12376 ] ]
  !gapprompt@gap>| !gapinput@for j in [1..4] do Display(Sum(List([1..4], i -> f[j][i]*DesignQPolynomial(D)(i-1))));|
  od;
  [ 1, 0, 0, 0 ]
  [ 0, 1, 0, 0 ]
  [ 0, 0, 1, 0 ]
  [ 0, 0, 0, 1 ]
\end{Verbatim}
 }

 }

 
\section{\textcolor{Chapter }{Designs with an Angle Set}}\logpage{[ 4, 3, 0 ]}
\hyperdef{L}{X87DA51657F7EC947}{}
{
  We can compute a number of properties of a design once the angle set is known. 

\subsection{\textcolor{Chapter }{IsDesignWithAngleSet}}
\logpage{[ 4, 3, 1 ]}\nobreak
\hyperdef{L}{X86A445CB8396ADAE}{}
{\noindent\textcolor{FuncColor}{$\triangleright$\enspace\texttt{IsDesignWithAngleSet\index{IsDesignWithAngleSet@\texttt{IsDesignWithAngleSet}}
\label{IsDesignWithAngleSet}
}\hfill{\scriptsize (filter)}}\\


This filter identifies the design as equipped with an angle set.}

 

\subsection{\textcolor{Chapter }{DesignAddAngleSet}}
\logpage{[ 4, 3, 2 ]}\nobreak
\hyperdef{L}{X7CF586B78445B681}{}
{\noindent\textcolor{FuncColor}{$\triangleright$\enspace\texttt{DesignAddAngleSet({\mdseries\slshape D, A})\index{DesignAddAngleSet@\texttt{DesignAddAngleSet}}
\label{DesignAddAngleSet}
}\hfill{\scriptsize (operation)}}\\


For a design \mbox{\texttt{\mdseries\slshape D}} without an angle set, records the angle set \mbox{\texttt{\mdseries\slshape A}} as an attribute \texttt{DesignAngleSet}. 
\begin{Verbatim}[commandchars=!@|,fontsize=\small,frame=single,label=Example]
  !gapprompt@gap>| !gapinput@D := DesignByJordanParameters(4,4);|
  <design with rank 4 and degree 4>
  !gapprompt@gap>| !gapinput@DesignAddAngleSet(D, [1/3,1/9]);|
  <design with rank 4, degree 4, and angle set [ 1/9, 1/3 ]>
  !gapprompt@gap>| !gapinput@DesignAngleSet(D);|
  [ 1/9, 1/3 ]
\end{Verbatim}
 }

 

\subsection{\textcolor{Chapter }{DesignByAngleSet}}
\logpage{[ 4, 3, 3 ]}\nobreak
\hyperdef{L}{X7A9BAE857BD9C7AB}{}
{\noindent\textcolor{FuncColor}{$\triangleright$\enspace\texttt{DesignByAngleSet({\mdseries\slshape rank, degree, A})\index{DesignByAngleSet@\texttt{DesignByAngleSet}}
\label{DesignByAngleSet}
}\hfill{\scriptsize (function)}}\\


Constructs a new design with Jordan rank and degree given by \mbox{\texttt{\mdseries\slshape rank}} and \mbox{\texttt{\mdseries\slshape degree}}, with angle set \mbox{\texttt{\mdseries\slshape A}}. 
\begin{Verbatim}[commandchars=!@|,fontsize=\small,frame=single,label=Example]
  !gapprompt@gap>| !gapinput@D := DesignByAngleSet(4, 4, [1/3, 1/9]);|
  <design with rank 4, degree 4, and angle set [ 1/9, 1/3 ]>
  !gapprompt@gap>| !gapinput@DesignAngleSet(D);|
  [ 1/9, 1/3 ]
\end{Verbatim}
 }

 

\subsection{\textcolor{Chapter }{DesignNormalizedAnnihilatorPolynomial}}
\logpage{[ 4, 3, 4 ]}\nobreak
\hyperdef{L}{X7B61ECE08124EAFA}{}
{\noindent\textcolor{FuncColor}{$\triangleright$\enspace\texttt{DesignNormalizedAnnihilatorPolynomial({\mdseries\slshape D})\index{DesignNormalizedAnnihilatorPolynomial@\texttt{Design}\-\texttt{Normalized}\-\texttt{Annihilator}\-\texttt{Polynomial}}
\label{DesignNormalizedAnnihilatorPolynomial}
}\hfill{\scriptsize (attribute)}}\\


The normalized annihilator polynomial is defined for an angle set in \cite[p. 185]{bannai_algebraic_2021}. This polynomial is stored as an attribute of a design with an angle set. 
\begin{Verbatim}[commandchars=!@|,fontsize=\small,frame=single,label=Example]
  !gapprompt@gap>| !gapinput@D := DesignByAngleSet(4, 4, [1/3, 1/9]);|
  <design with rank 4, degree 4, and angle set [ 1/9, 1/3 ]>
  !gapprompt@gap>| !gapinput@DesignNormalizedAnnihilatorPolynomial(D);|
  [ 1/16, -3/4, 27/16 ]
\end{Verbatim}
 }

 

\subsection{\textcolor{Chapter }{DesignNormalizedIndicatorCoefficients}}
\logpage{[ 4, 3, 5 ]}\nobreak
\hyperdef{L}{X87A3F05881AE56B8}{}
{\noindent\textcolor{FuncColor}{$\triangleright$\enspace\texttt{DesignNormalizedIndicatorCoefficients({\mdseries\slshape D})\index{DesignNormalizedIndicatorCoefficients@\texttt{Design}\-\texttt{Normalized}\-\texttt{Indicator}\-\texttt{Coefficients}}
\label{DesignNormalizedIndicatorCoefficients}
}\hfill{\scriptsize (attribute)}}\\


The normalized indicator coefficients are the \texttt{DesignQPolynomial(\mbox{\texttt{\mdseries\slshape D}})}-expansion coefficients of \texttt{DesignNormalizedAnnihilatorPolynomial(\mbox{\texttt{\mdseries\slshape D}})}, discussed for the spherical case in \cite[p. 185]{bannai_algebraic_2021}. These coefficients are stored as an attribute of a design with an angle set. 
\begin{Verbatim}[commandchars=!@|,fontsize=\small,frame=single,label=Example]
  !gapprompt@gap>| !gapinput@D := DesignByAngleSet(4, 4, [1/3, 1/9]);|
  <design with rank 4, degree 4, and angle set [ 1/9, 1/3 ]>
  !gapprompt@gap>| !gapinput@f := DesignNormalizedIndicatorCoefficients(D);|
  [ 1/64, 7/960, 9/3520 ]
  !gapprompt@gap>| !gapinput@Sum(List([1..3], i -> f[i]*DesignQPolynomial(D)(i-1)));|
  [ 1/16, -3/4, 27/16 ]
  !gapprompt@gap>| !gapinput@DesignNormalizedAnnihilatorPolynomial(D);|
  [ 1/16, -3/4, 27/16 ]
\end{Verbatim}
 }

 

\subsection{\textcolor{Chapter }{IsDesignWithPositiveIndicatorCoefficients}}
\logpage{[ 4, 3, 6 ]}\nobreak
\hyperdef{L}{X846C917E8683C0DC}{}
{\noindent\textcolor{FuncColor}{$\triangleright$\enspace\texttt{IsDesignWithPositiveIndicatorCoefficients\index{IsDesignWithPositiveIndicatorCoefficients@\texttt{IsDesign}\-\texttt{With}\-\texttt{Positive}\-\texttt{Indicator}\-\texttt{Coefficients}}
\label{IsDesignWithPositiveIndicatorCoefficients}
}\hfill{\scriptsize (filter)}}\\


 This filter determins whether the normalized indicator coefficients of a
design are positive, which has significance for certain theorems about
designs. }

 

\subsection{\textcolor{Chapter }{DesignSpecialBound}}
\logpage{[ 4, 3, 7 ]}\nobreak
\hyperdef{L}{X84FB19E27F0BC1FD}{}
{\noindent\textcolor{FuncColor}{$\triangleright$\enspace\texttt{DesignSpecialBound({\mdseries\slshape D})\index{DesignSpecialBound@\texttt{DesignSpecialBound}}
\label{DesignSpecialBound}
}\hfill{\scriptsize (attribute)}}\\


 The special bound of a design satisfying \texttt{IsDesignWithPositiveIndicatorCoefficients} is the upper limit on the possible cardinality for the given angle set. 
\begin{Verbatim}[commandchars=!@|,fontsize=\small,frame=single,label=Example]
  !gapprompt@gap>| !gapinput@D := DesignByAngleSet(4, 4, [1/3,1/9]);|
  <design with rank 4, degree 4, and angle set [ 1/9, 1/3 ]>
  !gapprompt@gap>| !gapinput@IsDesignWithPositiveIndicatorCoefficients(D);|
  true
  !gapprompt@gap>| !gapinput@DesignSpecialBound(D);|
  64
\end{Verbatim}
 }

 }

 
\section{\textcolor{Chapter }{Designs with Cardinality and Angle Set}}\logpage{[ 4, 4, 0 ]}
\hyperdef{L}{X852F4CD17B4A6587}{}
{
  
\subsection{\textcolor{Chapter }{Some Filters}}\logpage{[ 4, 4, 1 ]}
\hyperdef{L}{X7EF1B08A7DFE03A2}{}
{
\noindent\textcolor{FuncColor}{$\triangleright$\enspace\texttt{IsDesignWithCardinality\index{IsDesignWithCardinality@\texttt{IsDesignWithCardinality}}
\label{IsDesignWithCardinality}
}\hfill{\scriptsize (filter)}}\\
\noindent\textcolor{FuncColor}{$\triangleright$\enspace\texttt{IsRegularSchemeDesign\index{IsRegularSchemeDesign@\texttt{IsRegularSchemeDesign}}
\label{IsRegularSchemeDesign}
}\hfill{\scriptsize (filter)}}\\
\noindent\textcolor{FuncColor}{$\triangleright$\enspace\texttt{IsSpecialBoundDesign\index{IsSpecialBoundDesign@\texttt{IsSpecialBoundDesign}}
\label{IsSpecialBoundDesign}
}\hfill{\scriptsize (filter)}}\\
\noindent\textcolor{FuncColor}{$\triangleright$\enspace\texttt{IsAssociationSchemeDesign\index{IsAssociationSchemeDesign@\texttt{IsAssociationSchemeDesign}}
\label{IsAssociationSchemeDesign}
}\hfill{\scriptsize (filter)}}\\
\noindent\textcolor{FuncColor}{$\triangleright$\enspace\texttt{IsTightDesign\index{IsTightDesign@\texttt{IsTightDesign}}
\label{IsTightDesign}
}\hfill{\scriptsize (filter)}}\\


 A design with cardinality has a specified number of points. Given a design
with $v$ points and angle set $A$, it is possible to compute the strength $t$ of a design and write $s$ as the size of set $A$. When a design satisfies $t >= s - 1 $ it admits a regular scheme. A design at the special bound satisfies $t >= s$. When a design satisfies $t >= 2s - 2$ it admits an association scheme. Finally, when a design satisfies $t = 2s - 1$ for $0$ in $A$ or $t = 2s$ otherwise, it is a tight design. }

 

\subsection{\textcolor{Chapter }{DesignCardinality}}
\logpage{[ 4, 4, 2 ]}\nobreak
\hyperdef{L}{X7BDCEB11786A3D98}{}
{\noindent\textcolor{FuncColor}{$\triangleright$\enspace\texttt{DesignCardinality({\mdseries\slshape D})\index{DesignCardinality@\texttt{DesignCardinality}}
\label{DesignCardinality}
}\hfill{\scriptsize (attribute)}}\\


 Returns the cardinality of design \mbox{\texttt{\mdseries\slshape D}} when that design satisfies \texttt{IsDesignWithCardinality}. }

 

\subsection{\textcolor{Chapter }{DesignAddCardinality}}
\logpage{[ 4, 4, 3 ]}\nobreak
\hyperdef{L}{X811DD7267E1D35EF}{}
{\noindent\textcolor{FuncColor}{$\triangleright$\enspace\texttt{DesignAddCardinality({\mdseries\slshape D, v})\index{DesignAddCardinality@\texttt{DesignAddCardinality}}
\label{DesignAddCardinality}
}\hfill{\scriptsize (function)}}\\


This function stores the the specified cardinality \mbox{\texttt{\mdseries\slshape v}} as attribute \texttt{DesignCardinality} of design \mbox{\texttt{\mdseries\slshape D}}. The method requires the \mbox{\texttt{\mdseries\slshape D}} satisfies \texttt{IsDesignWithAngleSet}. 
\begin{Verbatim}[commandchars=!@|,fontsize=\small,frame=single,label=Example]
  !gapprompt@gap>| !gapinput@D := DesignByAngleSet(4, 4, [1/3,1/9]);|
  <design with rank 4, degree 4, and angle set [ 1/9, 1/3 ]>
  !gapprompt@gap>| !gapinput@DesignSpecialBound(D);|
  64
  !gapprompt@gap>| !gapinput@DesignAddCardinality(D, 64);|
  <design with rank 4, degree 4, cardinality 64, and angle set [ 1/9, 1/3 ]>
  !gapprompt@gap>| !gapinput@IsSpecialBoundDesign(D);|
  true
  !gapprompt@gap>| !gapinput@DesignCardinality(D);|
  64
\end{Verbatim}
 }

 

\subsection{\textcolor{Chapter }{IsDesignWithStrength}}
\logpage{[ 4, 4, 4 ]}\nobreak
\hyperdef{L}{X84289ABC794FBB20}{}
{\noindent\textcolor{FuncColor}{$\triangleright$\enspace\texttt{IsDesignWithStrength\index{IsDesignWithStrength@\texttt{IsDesignWithStrength}}
\label{IsDesignWithStrength}
}\hfill{\scriptsize (filter)}}\\


 This filter identifies designs for which the attribute \texttt{DesignStrength} is known. }

 

\subsection{\textcolor{Chapter }{DesignStrength}}
\logpage{[ 4, 4, 5 ]}\nobreak
\hyperdef{L}{X87B0DDCE84D6C6AF}{}
{\noindent\textcolor{FuncColor}{$\triangleright$\enspace\texttt{DesignStrength({\mdseries\slshape D})\index{DesignStrength@\texttt{DesignStrength}}
\label{DesignStrength}
}\hfill{\scriptsize (attribute)}}\\


 For a design \mbox{\texttt{\mdseries\slshape D}} that satisfies \texttt{IsDesignWithPositiveIndicatorCoefficients}, \texttt{IsDesignWithCardinality}, and \texttt{IsSpecialBoundDesign}, we can compute the strength $t$ of the design using the normalized indicator coefficients. This allows us to
immediately determine whether the design also satisfies \texttt{IsTightDesign} or \texttt{IsAssociationSchemeDesign}. 
\begin{Verbatim}[commandchars=!@|,fontsize=\small,frame=single,label=Example]
  !gapprompt@gap>| !gapinput@D;|
  <design with rank 4, degree 4, cardinality 64, and angle set [ 1/9, 1/3 ]>
  !gapprompt@gap>| !gapinput@IsAssociationSchemeDesign(D);|
  false
  !gapprompt@gap>| !gapinput@DesignStrength(D);|
  2
  !gapprompt@gap>| !gapinput@D;|
  <2-design with rank 4, degree 4, cardinality 64, and angle set [ 1/9, 1/3 ]>
\end{Verbatim}
 }

 

\subsection{\textcolor{Chapter }{DesignAnnihilatorPolynomial}}
\logpage{[ 4, 4, 6 ]}\nobreak
\hyperdef{L}{X855AB0FB83B45450}{}
{\noindent\textcolor{FuncColor}{$\triangleright$\enspace\texttt{DesignAnnihilatorPolynomial({\mdseries\slshape D})\index{DesignAnnihilatorPolynomial@\texttt{DesignAnnihilatorPolynomial}}
\label{DesignAnnihilatorPolynomial}
}\hfill{\scriptsize (attribute)}}\\


 The annihilator polynomial for design \mbox{\texttt{\mdseries\slshape D}} is defined by multiplying the \texttt{DesignNormalizedAnnihilatorPolynomial(\mbox{\texttt{\mdseries\slshape D}})} by \texttt{ DesignCardinality(\mbox{\texttt{\mdseries\slshape D}})}. 
\begin{Verbatim}[commandchars=!@|,fontsize=\small,frame=single,label=Example]
  !gapprompt@gap>| !gapinput@D := DesignByAngleSet(4, 4, [1/3, 1/9]);; DesignAddCardinality(D, 64);; D;|
  <design with rank 4, degree 4, cardinality 64, and angle set [ 1/9, 1/3 ]>
  !gapprompt@gap>| !gapinput@DesignAnnihilatorPolynomial(D);|
  [ 4, -48, 108 ]
\end{Verbatim}
 }

 

\subsection{\textcolor{Chapter }{DesignIndicatorCoefficients}}
\logpage{[ 4, 4, 7 ]}\nobreak
\hyperdef{L}{X7998AC4383F4F4C1}{}
{\noindent\textcolor{FuncColor}{$\triangleright$\enspace\texttt{DesignIndicatorCoefficients({\mdseries\slshape D})\index{DesignIndicatorCoefficients@\texttt{DesignIndicatorCoefficients}}
\label{DesignIndicatorCoefficients}
}\hfill{\scriptsize (attribute)}}\\


 The indicator coefficients for design \mbox{\texttt{\mdseries\slshape D}} are defined by multiplying \texttt{ DesignNormalizedIndicatorCoefficients(\mbox{\texttt{\mdseries\slshape D}})} by \texttt{ DesignCardinality(\mbox{\texttt{\mdseries\slshape  D}})}. These indicator coefficients are often useful for directly determining the
strength of a design at the special bound. 
\begin{Verbatim}[commandchars=!@|,fontsize=\small,frame=single,label=Example]
  !gapprompt@gap>| !gapinput@D := DesignByAngleSet(4, 4, [1/3, 1/9]);; DesignAddCardinality(D, 64);; D;|
  <design with rank 4, degree 4, cardinality 64, and angle set [ 1/9, 1/3 ]>
  !gapprompt@gap>| !gapinput@DesignIndicatorCoefficients(D);|
  [ 1, 7/15, 9/55 ]
\end{Verbatim}
 }

 }

 
\section{\textcolor{Chapter }{Designs Admitting a Regular Scheme}}\logpage{[ 4, 5, 0 ]}
\hyperdef{L}{X814E031B82F35E16}{}
{
  

\subsection{\textcolor{Chapter }{DesignSubdegrees}}
\logpage{[ 4, 5, 1 ]}\nobreak
\hyperdef{L}{X7DC041C678636666}{}
{\noindent\textcolor{FuncColor}{$\triangleright$\enspace\texttt{DesignSubdegrees({\mdseries\slshape D})\index{DesignSubdegrees@\texttt{DesignSubdegrees}}
\label{DesignSubdegrees}
}\hfill{\scriptsize (attribute)}}\\


 For a design \mbox{\texttt{\mdseries\slshape D}} with cardinality and angle set that satisfies \texttt{IsRegularSchemeDesign}, namely $t >= s - 1$, we can compute the regular subdegrees as described in \cite[Theorem 3.2]{hoggar_t-designs_1992}. 
\begin{Verbatim}[commandchars=!@|,fontsize=\small,frame=single,label=Example]
  !gapprompt@gap>| !gapinput@D := DesignByAngleSet(4, 4, [1/3, 1/9]);; DesignAddCardinality(D, 64);; D;|
  <design with rank 4, degree 4, cardinality 64, and angle set [ 1/9, 1/3 ]>
  !gapprompt@gap>| !gapinput@DesignSubdegrees(D);|
  [ 27, 36 ] 
\end{Verbatim}
 }

 }

 
\section{\textcolor{Chapter }{Designs Admitting an Association Scheme}}\logpage{[ 4, 6, 0 ]}
\hyperdef{L}{X7EE9F8D97A51FBF9}{}
{
  When a design satisfies $t > = 2s - 2$ then it also admits an association scheme. We can use results given in \cite{hoggar_t-designs_1992} to determine the parameters of the corresponding association scheme. 

\subsection{\textcolor{Chapter }{DesignBoseMesnerAlgebra}}
\logpage{[ 4, 6, 1 ]}\nobreak
\hyperdef{L}{X819FB2F77F56F9C1}{}
{\noindent\textcolor{FuncColor}{$\triangleright$\enspace\texttt{DesignBoseMesnerAlgebra({\mdseries\slshape D})\index{DesignBoseMesnerAlgebra@\texttt{DesignBoseMesnerAlgebra}}
\label{DesignBoseMesnerAlgebra}
}\hfill{\scriptsize (attribute)}}\\


 For a design that satisfies \texttt{IsAssociationSchemeDesign}, we can define the corresponding Bose-Mesner algebra \cite[pp. 53-57]{bannai_algebraic_2021}. The canonical basis for this algebra corresponds to the adjacency matrices $A_i$, with the \texttt{s+1}-th basis vector corresponding to $A_0$. 
\begin{Verbatim}[commandchars=!@|,fontsize=\small,frame=single,label=Example]
  !gapprompt@gap>| !gapinput@D := DesignByAngleSet(4,4,[1/3,1/9]);; DesignAddCardinality(D, 64);; D;|
  <2-design with rank 4, degree 4, cardinality 64, and angle set [ 1/9, 1/3 ]>
  !gapprompt@gap>| !gapinput@B := DesignBoseMesnerAlgebra(D);|
  <algebra of dimension 3 over Rationals>
  !gapprompt@gap>| !gapinput@BasisVectors(CanonicalBasis(B));|
  [ A1, A2, A3 ]
  !gapprompt@gap>| !gapinput@One(B);|
  A3 
\end{Verbatim}
 }

 

\subsection{\textcolor{Chapter }{DesignBoseMesnerIdempotentBasis}}
\logpage{[ 4, 6, 2 ]}\nobreak
\hyperdef{L}{X8524AF307BAF0680}{}
{\noindent\textcolor{FuncColor}{$\triangleright$\enspace\texttt{DesignBoseMesnerIdempotentBasis({\mdseries\slshape D})\index{DesignBoseMesnerIdempotentBasis@\texttt{DesignBoseMesnerIdempotentBasis}}
\label{DesignBoseMesnerIdempotentBasis}
}\hfill{\scriptsize (attribute)}}\\


For a design that satisfies \texttt{IsAssociationSchemeDesign}, we can also define the idempotent basis of the corresponding Bose-Mesner
algebra \cite[pp. 53-57]{bannai_algebraic_2021}. 
\begin{Verbatim}[commandchars=!@|,fontsize=\small,frame=single,label=Example]
  !gapprompt@gap>| !gapinput@D := DesignByAngleSet(4,4,[1/3,1/9]);; DesignAddCardinality(D, 64);; D;|
  <2-design with rank 4, degree 4, cardinality 64, and angle set [ 1/9, 1/3 ]>
  !gapprompt@gap>| !gapinput@DesignBoseMesnerIdempotentBasis(D);|
  Basis( <algebra of dimension 3 over Rationals>, [ (-5/64)*A1+(3/64)*A2+(27/64)*A3,
    (1/16)*A1+(-1/16)*A2+(9/16)*A3, (1/64)*A1+(1/64)*A2+(1/64)*A3 ] )
  !gapprompt@gap>| !gapinput@List(last, x -> x^2 = x);|
  [ true, true, true ] 
\end{Verbatim}
 }

 

\subsection{\textcolor{Chapter }{DesignIntersectionNumbers}}
\logpage{[ 4, 6, 3 ]}\nobreak
\hyperdef{L}{X7D657DB680515A4C}{}
{\noindent\textcolor{FuncColor}{$\triangleright$\enspace\texttt{DesignIntersectionNumbers({\mdseries\slshape D})\index{DesignIntersectionNumbers@\texttt{DesignIntersectionNumbers}}
\label{DesignIntersectionNumbers}
}\hfill{\scriptsize (attribute)}}\\


 The intersection numbers $p^k_{i,j}$ are given by \texttt{DesignIntersectionNumbers(\mbox{\texttt{\mdseries\slshape D}})[k][i][j]}. These intersection numbers serve as the structure constants for the \texttt{DesignBoseMesnerAlgebra(\mbox{\texttt{\mdseries\slshape D}})}. Namely, $A_i A_j = \sum_{k = 1}^{s+1} p^{k}_{i,j} A_k$. 
\begin{Verbatim}[commandchars=!@|,fontsize=\small,frame=single,label=Example]
  !gapprompt@gap>| !gapinput@D := DesignByAngleSet(4,4,[1/3,1/9]);; DesignAddCardinality(D, 64);; D;|
  <2-design with rank 4, degree 4, cardinality 64, and angle set [ 1/9, 1/3 ]>
  !gapprompt@gap>| !gapinput@A := BasisVectors(Basis(DesignBoseMesnerAlgebra(D)));;|
  [ A1, A2, A3 ]
  !gapprompt@gap>| !gapinput@p := DesignIntersectionNumbers(D);;|
  !gapprompt@gap>| !gapinput@A[1]*A[2] = Sum(List([1..3]), k -> p[k][1][2]*A[k]);|
  true
\end{Verbatim}
 }

 

\subsection{\textcolor{Chapter }{DesignKreinNumbers}}
\logpage{[ 4, 6, 4 ]}\nobreak
\hyperdef{L}{X8590B283798CE233}{}
{\noindent\textcolor{FuncColor}{$\triangleright$\enspace\texttt{DesignKreinNumbers({\mdseries\slshape D})\index{DesignKreinNumbers@\texttt{DesignKreinNumbers}}
\label{DesignKreinNumbers}
}\hfill{\scriptsize (attribute)}}\\


 The Krein numbers $q^k_{i,j}$ are given by \texttt{ DesignKreinNumbers(\mbox{\texttt{\mdseries\slshape D}})[k][i][j]}. The Krein numbers serve as the structure constants for the \texttt{DesignBoseMesnerAlgebra(\mbox{\texttt{\mdseries\slshape D}})} in the idempotent basis given by \texttt{DesignBoseMesnerIdempotentBasis(\mbox{\texttt{\mdseries\slshape D}})} using the Hadamard matrix product $\circ$. Namely, for idempotent basis $E_i$ and Hadamard product $\circ$, we have $E_i \circ E_j = \sum_{k = 1}^{s+1} q^{k}_{i,j} E_k$. 
\begin{Verbatim}[commandchars=!@|,fontsize=\small,frame=single,label=Example]
  !gapprompt@gap>| !gapinput@D := DesignByAngleSet(4,4,[1/3,1/9]);; DesignAddCardinality(D, 64);; D;|
  <2-design with rank 4, degree 4, cardinality 64, and angle set [ 1/9, 1/3 ]>
  !gapprompt@gap>| !gapinput@q := DesignKreinNumbers(D);|
  [ [ [ 10, 16, 1 ], [ 16, 20, 0 ], [ 1, 0, 0 ] ],
    [ [ 12, 15, 0 ], [ 15, 20, 1 ], [ 0, 1, 0 ] ],
    [ [ 27, 0, 0 ], [ 0, 36, 0 ], [ 0, 0, 1 ] ] ]
\end{Verbatim}
 }

 

\subsection{\textcolor{Chapter }{DesignFirstEigenmatrix}}
\logpage{[ 4, 6, 5 ]}\nobreak
\hyperdef{L}{X79269F0C7F7BF57F}{}
{\noindent\textcolor{FuncColor}{$\triangleright$\enspace\texttt{DesignFirstEigenmatrix({\mdseries\slshape D})\index{DesignFirstEigenmatrix@\texttt{DesignFirstEigenmatrix}}
\label{DesignFirstEigenmatrix}
}\hfill{\scriptsize (attribute)}}\\


 As describe in \cite[p. 58]{bannai_algebraic_2021}, the first eigenmatrix of a Bose-Mesner algebra $P_i(j)$ defines the expansion of the adjacency matrix basis $A_i$ in terms of the idempotent basis $E_j$ as follows: $A_i = \sum_{j = 1}^{s+1} P_i(j) E_j $. This attribute returns the component $P_i(j)$ as \texttt{DesignFirstEigenmatrix(\mbox{\texttt{\mdseries\slshape D}})[i][j]}. 
\begin{Verbatim}[commandchars=!@|,fontsize=\small,frame=single,label=Example]
  !gapprompt@gap>| !gapinput@D := DesignByAngleSet(4,4,[1/3,1/9]);; DesignAddCardinality(D, 64);; D;|
  <2-design with rank 4, degree 4, cardinality 64, and angle set [ 1/9, 1/3 ]>
  !gapprompt@gap>| !gapinput@a := Basis(DesignBoseMesnerAlgebra(D));; e := DesignBoseMesnerIdempotentBasis(D);;|
  !gapprompt@gap>| !gapinput@List([1..3], i -> a[i] = Sum([1..3], j -> DesignFirstEigenmatrix(D)[i][j]*e[j]));|
  [ true, true, true ] 
\end{Verbatim}
 }

 

\subsection{\textcolor{Chapter }{DesignSecondEigenmatrix}}
\logpage{[ 4, 6, 6 ]}\nobreak
\hyperdef{L}{X7BA18016817E9A9D}{}
{\noindent\textcolor{FuncColor}{$\triangleright$\enspace\texttt{DesignSecondEigenmatrix({\mdseries\slshape D})\index{DesignSecondEigenmatrix@\texttt{DesignSecondEigenmatrix}}
\label{DesignSecondEigenmatrix}
}\hfill{\scriptsize (attribute)}}\\


 As describe in \cite[p. 58]{bannai_algebraic_2021}, the second eigenmatrix of a Bose-Mesner algebra $Q_i(j)$ defines the expansion of the idempotent basis $E_i$ in terms of the adjacency matrix basis $A_j$ as follows: $E_i = (1/v)\sum_{j = 1}^{s+1} Q_i(j) A_j $. This attribute returns the component $Q_i(j)$ as \texttt{DesignSecondEigenmatrix(\mbox{\texttt{\mdseries\slshape D}})[i][j]}. 
\begin{Verbatim}[commandchars=!@|,fontsize=\small,frame=single,label=Example]
  !gapprompt@gap>| !gapinput@D := DesignByAngleSet(4,4,[1/3,1/9]);; DesignAddCardinality(D, 64);; D;|
  <2-design with rank 4, degree 4, cardinality 64, and angle set [ 1/9, 1/3 ]>
  !gapprompt@gap>| !gapinput@a := Basis(DesignBoseMesnerAlgebra(D));; e := DesignBoseMesnerIdempotentBasis(D);;|
  !gapprompt@gap>| !gapinput@List([1..3], i -> e[i]*DesignCardinality(D) = Sum([1..3], j -> DesignSecondEigenmatrix(D)[i][j]*a[j]));|
  [ true, true, true ] 
  !gapprompt@gap>| !gapinput@DesignFirstEigenmatrix(D) = Inverse(DesignSecondEigenmatrix(D))*DesignCardinality(D);|
  true
\end{Verbatim}
 }

 

\subsection{\textcolor{Chapter }{DesignMultiplicities}}
\logpage{[ 4, 6, 7 ]}\nobreak
\hyperdef{L}{X797F3A50809DAB1C}{}
{\noindent\textcolor{FuncColor}{$\triangleright$\enspace\texttt{DesignMultiplicities({\mdseries\slshape D})\index{DesignMultiplicities@\texttt{DesignMultiplicities}}
\label{DesignMultiplicities}
}\hfill{\scriptsize (attribute)}}\\


 As describe in \cite[pp. 58-59]{bannai_algebraic_2021}, the design multiplicy $m_i$ is defined as the dimension of the space that idempotent matrix $E_i$ projects onto, or $m_i = trace(E_i)$. We also have $m_i = Q_i(s+1)$. 
\begin{Verbatim}[commandchars=!@|,fontsize=\small,frame=single,label=Example]
  !gapprompt@gap>| !gapinput@D := DesignByAngleSet(4,4,[1/3,1/9]);; DesignAddCardinality(D, 64);; D;|
  <2-design with rank 4, degree 4, cardinality 64, and angle set [ 1/9, 1/3 ]>
  !gapprompt@gap>| !gapinput@DesignMultiplicities(D);|
  [ 27, 36, 1 ]
\end{Verbatim}
 }

 

\subsection{\textcolor{Chapter }{DesignValencies}}
\logpage{[ 4, 6, 8 ]}\nobreak
\hyperdef{L}{X7E7012EB7BFE889D}{}
{\noindent\textcolor{FuncColor}{$\triangleright$\enspace\texttt{DesignValencies({\mdseries\slshape D})\index{DesignValencies@\texttt{DesignValencies}}
\label{DesignValencies}
}\hfill{\scriptsize (attribute)}}\\


 As describe in \cite[pp. 55, 59]{bannai_algebraic_2021}, the design valency $k_i$ is defined as the fixed number of $i$-associates of any element in the association scheme (also known as the
subdegree). We also have $k_i = P_i(s+1)$. 
\begin{Verbatim}[commandchars=!@|,fontsize=\small,frame=single,label=Example]
  !gapprompt@gap>| !gapinput@D := DesignByAngleSet(4,4,[1/3,1/9]);; DesignAddCardinality(D, 64);; D;|
  <2-design with rank 4, degree 4, cardinality 64, and angle set [ 1/9, 1/3 ]>
  !gapprompt@gap>| !gapinput@DesignValencies(D);|
  [ 27, 36, 1 ]
\end{Verbatim}
 }

 

\subsection{\textcolor{Chapter }{DesignReducedAdjacencyMatrices}}
\logpage{[ 4, 6, 9 ]}\nobreak
\hyperdef{L}{X849A9E5B79E8C3EE}{}
{\noindent\textcolor{FuncColor}{$\triangleright$\enspace\texttt{DesignReducedAdjacencyMatrices({\mdseries\slshape D})\index{DesignReducedAdjacencyMatrices@\texttt{DesignReducedAdjacencyMatrices}}
\label{DesignReducedAdjacencyMatrices}
}\hfill{\scriptsize (attribute)}}\\


 As defined in \cite[p. 201]{cameron_designs_1991}, the reduced adjacency matrices multiply with the same structure constants as
the adjacency matrices, which allows for a simpler construction of an algebra
isomorphic to the Bose-Mesner algebra. The matrices \texttt{DesignReducedAdjacencyMatrices(\mbox{\texttt{\mdseries\slshape D}})} are used to construct \texttt{DesignBoseMesnerAlgebra(\mbox{\texttt{\mdseries\slshape D}})}. }

 }

 
\section{\textcolor{Chapter }{Examples}}\logpage{[ 4, 7, 0 ]}
\hyperdef{L}{X7A489A5D79DA9E5C}{}
{
  The following tight projective t-designs are identified in \cite[Examples 1-11]{hoggar_t-designs_1982}. 
\begin{Verbatim}[commandchars=!@|,fontsize=\small,frame=single,label=Example]
  !gapprompt@gap>| !gapinput@DesignByAngleSet(2, 1, [0,1/2]);; DesignAddCardinality(last, DesignSpecialBound(last));|
  <Tight 3-design with rank 2, degree 1, cardinality 4, and angle set [ 0, 1/2 ]>
  !gapprompt@gap>| !gapinput@DesignByAngleSet(2, 2, [0,1/2]);; DesignAddCardinality(last, DesignSpecialBound(last));|
  <Tight 3-design with rank 2, degree 2, cardinality 6, and angle set [ 0, 1/2 ]>
  !gapprompt@gap>| !gapinput@DesignByAngleSet(2, 4, [0,1/2]);; DesignAddCardinality(last, DesignSpecialBound(last));|
  <Tight 3-design with rank 2, degree 4, cardinality 10, and angle set [ 0, 1/2 ]>
  !gapprompt@gap>| !gapinput@DesignByAngleSet(2, 8, [0,1/2]);; DesignAddCardinality(last, DesignSpecialBound(last));|
  <Tight 3-design with rank 2, degree 8, cardinality 18, and angle set [ 0, 1/2 ]>
  !gapprompt@gap>| !gapinput@DesignByAngleSet(3, 2, [1/4]);; DesignAddCardinality(last, DesignSpecialBound(last));|
  <Tight 2-design with rank 3, degree 2, cardinality 9, and angle set [ 1/4 ]>
  !gapprompt@gap>| !gapinput@DesignByAngleSet(4, 2, [0,1/3]);; DesignAddCardinality(last, DesignSpecialBound(last));|
  <Tight 3-design with rank 4, degree 2, cardinality 40, and angle set [ 0, 1/3 ]>
  !gapprompt@gap>| !gapinput@DesignByAngleSet(6, 2, [0,1/4]);; DesignAddCardinality(last, DesignSpecialBound(last));|
  <Tight 3-design with rank 6, degree 2, cardinality 126, and angle set [ 0, 1/4 ]>
  !gapprompt@gap>| !gapinput@DesignByAngleSet(8, 2, [1/9]);; DesignAddCardinality(last, DesignSpecialBound(last));|
  <Tight 2-design with rank 8, degree 2, cardinality 64, and angle set [ 1/9 ]>
  !gapprompt@gap>| !gapinput@DesignByAngleSet(5, 4, [0,1/4]);; DesignAddCardinality(last, DesignSpecialBound(last));|
  <Tight 3-design with rank 5, degree 4, cardinality 165, and angle set [ 0, 1/4 ]>
  !gapprompt@gap>| !gapinput@DesignByAngleSet(3, 8, [0,1/4,1/2]);; DesignAddCardinality(last, DesignSpecialBound(las|
  t));
  <Tight 5-design with rank 3, degree 8, cardinality 819, and angle set [ 0, 1/4, 1/2 ]>
  !gapprompt@gap>| !gapinput@DesignByAngleSet(24, 1, [0,1/16,1/4]);; DesignAddCardinality(last, DesignSpecialBound(l|
  ast));
  <Tight 5-design with rank 24, degree 1, cardinality 98280, and angle set [ 0, 1/16, 1/4 ]>
\end{Verbatim}
 An additional icosahedron projective example is identified in \cite{lyubich_tight_2009}. 
\begin{Verbatim}[commandchars=!@|,fontsize=\small,frame=single,label=Example]
  !gapprompt@gap>| !gapinput@DesignByAngleSet(2, 2, [ 0, (5-Sqrt(5))/10, (5+Sqrt(5))/10 ]);; DesignAddCardinality(last, DesignSpecialBound(last));|
  <Tight 5-design with rank 2, degree 2, cardinality 12, and angle set
  [ 0, -3/5*E(5)-2/5*E(5)^2-2/5*E(5)^3-3/5*E(5)^4,
    -2/5*E(5)-3/5*E(5)^2-3/5*E(5)^3-2/5*E(5)^4 ]>
\end{Verbatim}
 The Leech lattice short vector design and several other tight spherical
designs are given below: 
\begin{Verbatim}[commandchars=!@|,fontsize=\small,frame=single,label=Example]
  !gapprompt@gap>| !gapinput@DesignByAngleSet(2, 23, [ 0, 1/4, 3/8, 1/2, 5/8, 3/4 ]);; DesignAddCardinality(last, De|
  signSpecialBound(last));
  <Tight 11-design with rank 2, degree 23, cardinality 196560, and angle set
  [ 0, 1/4, 3/8, 1/2, 5/8, 3/4 ]>
  !gapprompt@gap>| !gapinput@DesignByAngleSet(2, 5, [ 1/4, 5/8 ]);; DesignAddCardinality(last, DesignSpecialBound(la|
  st));
  <Tight 4-design with rank 2, degree 5, cardinality 27, and angle set [ 1/4, 5/8 ]>
  !gapprompt@gap>| !gapinput@DesignByAngleSet(2, 6, [0,1/3,2/3]);; DesignAddCardinality(last, DesignSpecialBound(las|
  t));
  <Tight 5-design with rank 2, degree 6, cardinality 56, and angle set [ 0, 1/3, 2/3 ]>
  !gapprompt@gap>| !gapinput@DesignByAngleSet(2, 21, [3/8, 7/12]);; DesignAddCardinality(last, DesignSpecialBound(la|
  st));
  <Tight 4-design with rank 2, degree 21, cardinality 275, and angle set [ 3/8, 7/12 ]>
  !gapprompt@gap>| !gapinput@DesignByAngleSet(2, 22, [0,2/5,3/5]);; DesignAddCardinality(last, DesignSpecialBound(la|
  st));
  <Tight 5-design with rank 2, degree 22, cardinality 552, and angle set [ 0, 2/5, 3/5 ]>
  !gapprompt@gap>| !gapinput@DesignByAngleSet(2, 7, [0,1/4,1/2,3/4]);; DesignAddCardinality(last, DesignSpecialBound|
  (last));
  <Tight 7-design with rank 2, degree 7, cardinality 240, and angle set [ 0, 1/4, 1/2, 3/4
   ]>
  !gapprompt@gap>| !gapinput@DesignByAngleSet(2, 22, [0,1/3,1/2,2/3]);; DesignAddCardinality(last, DesignSpecialBoun|
  d(last));
  <Tight 7-design with rank 2, degree 22, cardinality 4600, and angle set
  [ 0, 1/3, 1/2, 2/3 ]>
\end{Verbatim}
 Some projective designs meeting the special bound are given in \cite[Examples 1-11]{hoggar_t-designs_1982}: 
\begin{Verbatim}[commandchars=!@|,fontsize=\small,frame=single,label=Example]
  !gapprompt@gap>| !gapinput@DesignByAngleSet(4, 4, [0,1/4,1/2]);; DesignAddCardinality(last, DesignSpecialBound(las|
  t));
  <3-design with rank 4, degree 4, cardinality 180, and angle set [ 0, 1/4, 1/2 ]>
  !gapprompt@gap>| !gapinput@DesignByAngleSet(3, 2, [0,1/3]);; DesignAddCardinality(last, DesignSpecialBound(last));|
  <2-design with rank 3, degree 2, cardinality 12, and angle set [ 0, 1/3 ]>
  !gapprompt@gap>| !gapinput@DesignByAngleSet(5, 2, [0,1/4]);; DesignAddCardinality(last, DesignSpecialBound(last));|
  <2-design with rank 5, degree 2, cardinality 45, and angle set [ 0, 1/4 ]>
  !gapprompt@gap>| !gapinput@DesignByAngleSet(9, 2, [0,1/9]);; DesignAddCardinality(last, DesignSpecialBound(last));|
  <2-design with rank 9, degree 2, cardinality 90, and angle set [ 0, 1/9 ]>
  !gapprompt@gap>| !gapinput@DesignByAngleSet(28, 2, [0,1/16]);; DesignAddCardinality(last, DesignSpecialBound(last)|
  );
  <2-design with rank 28, degree 2, cardinality 4060, and angle set [ 0, 1/16 ]>
  !gapprompt@gap>| !gapinput@DesignByAngleSet(4, 4, [0,1/4]);; DesignAddCardinality(last, DesignSpecialBound(last));|
  <2-design with rank 4, degree 4, cardinality 36, and angle set [ 0, 1/4 ]>
  !gapprompt@gap>| !gapinput@DesignByAngleSet(4, 4, [1/9,1/3]);; DesignAddCardinality(last, DesignSpecialBound(last)|
  );
  <2-design with rank 4, degree 4, cardinality 64, and angle set [ 1/9, 1/3 ]>
  !gapprompt@gap>| !gapinput@DesignByAngleSet(16, 1, [0,1/9]);; DesignAddCardinality(last, DesignSpecialBound(last))|
  ;
  <2-design with rank 16, degree 1, cardinality 256, and angle set [ 0, 1/9 ]>
  !gapprompt@gap>| !gapinput@DesignByAngleSet(4, 2, [0,1/4,1/2]);; DesignAddCardinality(last, DesignSpecialBound(las|
  t));
  <3-design with rank 4, degree 2, cardinality 60, and angle set [ 0, 1/4, 1/2 ]>
  !gapprompt@gap>| !gapinput@DesignByAngleSet(16, 1, [0,1/16,1/4]);; DesignAddCardinality(last, DesignSpecialBound(l|
  ast));
  <3-design with rank 16, degree 1, cardinality 2160, and angle set [ 0, 1/16, 1/4 ]>
  !gapprompt@gap>| !gapinput@DesignByAngleSet(3, 4, [0,1/4,1/2]);; DesignAddCardinality(last, DesignSpecialBound(las|
  t));
  <3-design with rank 3, degree 4, cardinality 63, and angle set [ 0, 1/4, 1/2 ]>
  !gapprompt@gap>| !gapinput@DesignByAngleSet(3, 4, [0,1/4,1/2,(3+Sqrt(5))/8, (3-Sqrt(5))/8]);; DesignAddCardinality|
  (last, DesignSpecialBound(last));
  <5-design with rank 3, degree 4, cardinality 315, and angle set
  [ 0, 1/4, 1/2, -1/2*E(5)-1/4*E(5)^2-1/4*E(5)^3-1/2*E(5)^4,
    -1/4*E(5)-1/2*E(5)^2-1/2*E(5)^3-1/4*E(5)^4 ]>
  !gapprompt@gap>| !gapinput@DesignByAngleSet(12, 2, [0,1/3,1/4,1/12]);; DesignAddCardinality(last, DesignSpecialBou|
  nd(last));
  <5-design with rank 12, degree 2, cardinality 32760, and angle set [ 0, 1/12, 1/4, 1/3 ]>
\end{Verbatim}
 Two important designs related to the $H_4$ Weyl group are as follows: 
\begin{Verbatim}[commandchars=!@|,fontsize=\small,frame=single,label=Example]
  !gapprompt@gap>| !gapinput@A := [ 0, 1/4, 1/2, 3/4, (5-Sqrt(5))/8, (5+Sqrt(5))/8, (3-Sqrt(5))/8, (3+Sqrt(5))/8 ];;|
  !gapprompt@gap>| !gapinput@D := DesignByAngleSet(2, 3, A);; DesignAddCardinality(D, DesignSpecialBound(D));|
  <11-design with rank 2, degree 3, cardinality 120, and angle set [ 0, 1/4, 1/2, 3/4, -3/4*E(5)-1/2*E(5)^2-1/2*E(5)^3-3/4*E(5)^4, -1/2*E(5)-3/4*E(5)^2-3/4*E(5)^3-1/2*E(5)^4, -1/2*E(5)-1/4*E(5)^2-1/4*E(5)^3-1/2*E(5)^4, -1/4*E(5)-1/2*E(5)^2-1/2*E(5)^3-1/4*E(5)^4 ]>
  !gapprompt@gap>| !gapinput@A := [ 0, 1/4, (3-Sqrt(5))/8, (3+Sqrt(5))/8 ];;|
  !gapprompt@gap>| !gapinput@D := DesignByAngleSet(4, 1, A);; DesignAddCardinality(D, DesignSpecialBound(D));|
  <5-design with rank 4, degree 1, cardinality 60, and angle set
  [ 0, 1/4, -1/2*E(5)-1/4*E(5)^2-1/4*E(5)^3-1/2*E(5)^4,
    -1/4*E(5)-1/2*E(5)^2-1/2*E(5)^3-1/4*E(5)^4 ]> 
\end{Verbatim}
 }

 }

 
\chapter{\textcolor{Chapter }{Octonion Lattice Constructions}}\logpage{[ 5, 0, 0 ]}
\hyperdef{L}{X7F6AA3C97E706F4F}{}
{
  In what follows let \mbox{\texttt{\mdseries\slshape L}} be a free left {\ensuremath{\mathbb Z}}-module that satisfies \texttt{IsOctonionLattice}. 
\section{\textcolor{Chapter }{Gram Matrix Filters}}\logpage{[ 5, 1, 0 ]}
\hyperdef{L}{X87267018841AF980}{}
{
  

\subsection{\textcolor{Chapter }{IsLeechLatticeGramMatrix}}
\logpage{[ 5, 1, 1 ]}\nobreak
\hyperdef{L}{X86B3BD9C84639A7E}{}
{\noindent\textcolor{FuncColor}{$\triangleright$\enspace\texttt{IsLeechLatticeGramMatrix({\mdseries\slshape G})\index{IsLeechLatticeGramMatrix@\texttt{IsLeechLatticeGramMatrix}}
\label{IsLeechLatticeGramMatrix}
}\hfill{\scriptsize (function)}}\\


 This function returns \texttt{true} when \mbox{\texttt{\mdseries\slshape G}} is a Gram matrix of a Leech lattice and \texttt{false} otherwise. Specifically, this function confirms that the lattice defined by \mbox{\texttt{\mdseries\slshape G}} is unimodular with shortest vectors of length at least 4. }

 

\subsection{\textcolor{Chapter }{IsGossetLatticeGramMatrix}}
\logpage{[ 5, 1, 2 ]}\nobreak
\hyperdef{L}{X78F07E967C6779FF}{}
{\noindent\textcolor{FuncColor}{$\triangleright$\enspace\texttt{IsGossetLatticeGramMatrix({\mdseries\slshape G})\index{IsGossetLatticeGramMatrix@\texttt{IsGossetLatticeGramMatrix}}
\label{IsGossetLatticeGramMatrix}
}\hfill{\scriptsize (function)}}\\


 This function returns \texttt{true} when \mbox{\texttt{\mdseries\slshape G}} is a Gram matrix of a Gosset ($E_8$) lattice and \texttt{false} otherwise. Specifically, this function confirms that the lattice defined by \mbox{\texttt{\mdseries\slshape G}} is unimodular with shortest vectors of length at least 2. }

 

\subsection{\textcolor{Chapter }{IsOctonionLattice}}
\logpage{[ 5, 1, 3 ]}\nobreak
\hyperdef{L}{X7C043B907EC1FE89}{}
{\noindent\textcolor{FuncColor}{$\triangleright$\enspace\texttt{IsOctonionLattice\index{IsOctonionLattice@\texttt{IsOctonionLattice}}
\label{IsOctonionLattice}
}\hfill{\scriptsize (filter)}}\\


This is a subcategory of \texttt{IsFreeLeftModule} used below to construct octonion lattices with an inner product defined via an
octonion gram matrix.}

 }

 
\section{\textcolor{Chapter }{Octonion Lattice Constructions}}\logpage{[ 5, 2, 0 ]}
\hyperdef{L}{X7F6AA3C97E706F4F}{}
{
  

\subsection{\textcolor{Chapter }{OctonionLatticeByGenerators}}
\logpage{[ 5, 2, 1 ]}\nobreak
\hyperdef{L}{X7FCBD0FF7D8C0C19}{}
{\noindent\textcolor{FuncColor}{$\triangleright$\enspace\texttt{OctonionLatticeByGenerators({\mdseries\slshape gens[, g]})\index{OctonionLatticeByGenerators@\texttt{OctonionLatticeByGenerators}}
\label{OctonionLatticeByGenerators}
}\hfill{\scriptsize (function)}}\\


 For \mbox{\texttt{\mdseries\slshape gens}} a list of octonion vectors, so that \mbox{\texttt{\mdseries\slshape gens}} satisfies \texttt{IsOctonionCollColl}, this function constructs a free left {\ensuremath{\mathbb Z}}-module that
satisfies \texttt{IsOctonionLattice}. The attribute \texttt{LeftActingDomain} is set to \texttt{Integers} and the input \mbox{\texttt{\mdseries\slshape gens}} is stored as the attribute \texttt{GeneratorsOfLeftOperatorAdditiveGroup}. The inner product on the lattice is defined by optional argument \mbox{\texttt{\mdseries\slshape g}}, which is an octonion square matrix that defaults to the identity matrix. For \mbox{\texttt{\mdseries\slshape x,y}} octonion vectors in the lattice, the inner product is computed as \texttt{ ScalarProduct(\mbox{\texttt{\mdseries\slshape L}}, \mbox{\texttt{\mdseries\slshape x}}, \mbox{\texttt{\mdseries\slshape y}}) = Trace(\mbox{\texttt{\mdseries\slshape x}}*\mbox{\texttt{\mdseries\slshape g}}*ComplexConjugate(\mbox{\texttt{\mdseries\slshape y}}))}. 
\begin{Verbatim}[commandchars=!@|,fontsize=\small,frame=single,label=Example]
  !gapprompt@gap>| !gapinput@O := OctonionArithmetic(Integers);; gens := Concatenation(List(Basis(O), x -> x*IdentityMat(3)));;|
  !gapprompt@gap>| !gapinput@O3 := OctonionLatticeByGenerators(gens);|
  <free left module over Integers, with 24 generators>
\end{Verbatim}
 }

 }

 
\section{\textcolor{Chapter }{Octonion Lattice Attributes}}\logpage{[ 5, 3, 0 ]}
\hyperdef{L}{X786A725B7ADE7BDE}{}
{
  

\subsection{\textcolor{Chapter }{UnderlyingOctonionRing}}
\logpage{[ 5, 3, 1 ]}\nobreak
\hyperdef{L}{X781F36A17CD9FDA6}{}
{\noindent\textcolor{FuncColor}{$\triangleright$\enspace\texttt{UnderlyingOctonionRing({\mdseries\slshape L})\index{UnderlyingOctonionRing@\texttt{UnderlyingOctonionRing}}
\label{UnderlyingOctonionRing}
}\hfill{\scriptsize (attribute)}}\\


 This attribute stores the octonion algebra containing the octonion
coefficients of the generating vectors, stored as \texttt{GeneratorsOfLeftOperatorAdditiveGroup(\mbox{\texttt{\mdseries\slshape L}})}. }

 

\subsection{\textcolor{Chapter }{OctonionGramMatrix}}
\logpage{[ 5, 3, 2 ]}\nobreak
\hyperdef{L}{X8120692484549A5B}{}
{\noindent\textcolor{FuncColor}{$\triangleright$\enspace\texttt{OctonionGramMatrix({\mdseries\slshape L})\index{OctonionGramMatrix@\texttt{OctonionGramMatrix}}
\label{OctonionGramMatrix}
}\hfill{\scriptsize (attribute)}}\\


 This attribute stores the optional argument \mbox{\texttt{\mdseries\slshape g}} of \texttt{OctonionLatticeByGenerators(\mbox{\texttt{\mdseries\slshape gens}} [,\mbox{\texttt{\mdseries\slshape g}}])}. This attribute stores the octonion matrix used to calculate the inner
product on the lattice via \texttt{Trace(\mbox{\texttt{\mdseries\slshape x}}*\mbox{\texttt{\mdseries\slshape g}}*ComplexConjugate(\mbox{\texttt{\mdseries\slshape y}}))}. The default value of this attribute is the identity matrix. }

 

\subsection{\textcolor{Chapter }{Dimension}}
\logpage{[ 5, 3, 3 ]}\nobreak
\hyperdef{L}{X7E6926C6850E7C4E}{}
{\noindent\textcolor{FuncColor}{$\triangleright$\enspace\texttt{Dimension({\mdseries\slshape L})\index{Dimension@\texttt{Dimension}}
\label{Dimension}
}\hfill{\scriptsize (attribute)}}\\


 For \mbox{\texttt{\mdseries\slshape L}} satisfying \texttt{IsOctonionLattice} these attributes determine the lattice rank, which is equivalent to the
lattice dimension. The value is computed by determining \texttt{Rank(GeneratorsAsCoefficients(\mbox{\texttt{\mdseries\slshape L}}))}. }

 

\subsection{\textcolor{Chapter }{GeneratorsAsCoefficients}}
\logpage{[ 5, 3, 4 ]}\nobreak
\hyperdef{L}{X79113ABC7A39CFFB}{}
{\noindent\textcolor{FuncColor}{$\triangleright$\enspace\texttt{GeneratorsAsCoefficients({\mdseries\slshape L})\index{GeneratorsAsCoefficients@\texttt{GeneratorsAsCoefficients}}
\label{GeneratorsAsCoefficients}
}\hfill{\scriptsize (attribute)}}\\


 This attributes converts the lattice generators, stored as \texttt{GeneratorsOfLeftOperatorAdditiveGroup(\mbox{\texttt{\mdseries\slshape L}})}, into a list of coefficients. For each generating vector \mbox{\texttt{\mdseries\slshape x}}, the coefficient list \texttt{OctonionToRealVector(CanonicalBasis(UnderlyingOctonionRing(\mbox{\texttt{\mdseries\slshape L}})), \mbox{\texttt{\mdseries\slshape x}})} is added to the list \texttt{ GeneratorsAsCoefficients(\mbox{\texttt{\mdseries\slshape L}})}. }

 

\subsection{\textcolor{Chapter }{LLLReducedBasisCoefficients}}
\logpage{[ 5, 3, 5 ]}\nobreak
\hyperdef{L}{X80B8907778F550EE}{}
{\noindent\textcolor{FuncColor}{$\triangleright$\enspace\texttt{LLLReducedBasisCoefficients({\mdseries\slshape L})\index{LLLReducedBasisCoefficients@\texttt{LLLReducedBasisCoefficients}}
\label{LLLReducedBasisCoefficients}
}\hfill{\scriptsize (attribute)}}\\


 This attribute stores the result of \texttt{LLLReducedBasis(\mbox{\texttt{\mdseries\slshape L}}, GeneratorsAsCoefficients(\mbox{\texttt{\mdseries\slshape L}})).basis}. This provides a set of basis vectors as coefficients for \mbox{\texttt{\mdseries\slshape L}}, since there is no test to ensure that \texttt{GeneratorsOfLeftOperatorAdditiveGroup} form a {\ensuremath{\mathbb Z}}-module basis. The \texttt{LLLReducedBasis} operation is conducted with reference to \texttt{ScalarProduct(\mbox{\texttt{\mdseries\slshape L}}, \mbox{\texttt{\mdseries\slshape x}}, \mbox{\texttt{\mdseries\slshape y}})}, which is defined }

 

\subsection{\textcolor{Chapter }{GramMatrix (GramMatrixLattice)}}
\logpage{[ 5, 3, 6 ]}\nobreak
\hyperdef{L}{X8728E0997EFD3E63}{}
{\noindent\textcolor{FuncColor}{$\triangleright$\enspace\texttt{GramMatrix({\mdseries\slshape L})\index{GramMatrix@\texttt{GramMatrix}!GramMatrixLattice}
\label{GramMatrix:GramMatrixLattice}
}\hfill{\scriptsize (attribute)}}\\


 This attribute stores the Gram matrix of vectors \texttt{LLLReducedBasisCoefficients(\mbox{\texttt{\mdseries\slshape L}})} relative to \texttt{ScalarProduct(\mbox{\texttt{\mdseries\slshape L}}, \mbox{\texttt{\mdseries\slshape x}}, \mbox{\texttt{\mdseries\slshape y}})}. }

 

\subsection{\textcolor{Chapter }{TotallyIsotropicCode}}
\logpage{[ 5, 3, 7 ]}\nobreak
\hyperdef{L}{X8127307B7E5A01CC}{}
{\noindent\textcolor{FuncColor}{$\triangleright$\enspace\texttt{TotallyIsotropicCode({\mdseries\slshape L})\index{TotallyIsotropicCode@\texttt{TotallyIsotropicCode}}
\label{TotallyIsotropicCode}
}\hfill{\scriptsize (attribute)}}\\


 This attribute stores the vectorspace over \texttt{GF(2)} generated by the vectors \texttt{LLLReducedBasisCoefficients(\mbox{\texttt{\mdseries\slshape L}})} multiplied by \texttt{Z(2)} (see \cite{lepowsky_e8-approach_1982} for more details). }

 
\subsection{\textcolor{Chapter }{Lattice Basis}}\logpage{[ 5, 3, 8 ]}
\hyperdef{L}{X825D41AE7A411640}{}
{
\noindent\textcolor{FuncColor}{$\triangleright$\enspace\texttt{Basis({\mdseries\slshape L})\index{Basis@\texttt{Basis}}
\label{Basis}
}\hfill{\scriptsize (attribute)}}\\
\noindent\textcolor{FuncColor}{$\triangleright$\enspace\texttt{CanonicalBasis({\mdseries\slshape L})\index{CanonicalBasis@\texttt{CanonicalBasis}}
\label{CanonicalBasis}
}\hfill{\scriptsize (attribute)}}\\
\noindent\textcolor{FuncColor}{$\triangleright$\enspace\texttt{BasisVectors({\mdseries\slshape B})\index{BasisVectors@\texttt{BasisVectors}}
\label{BasisVectors}
}\hfill{\scriptsize (attribute)}}\\
\noindent\textcolor{FuncColor}{$\triangleright$\enspace\texttt{IsOctonionLatticeBasis\index{IsOctonionLatticeBasis@\texttt{IsOctonionLatticeBasis}}
\label{IsOctonionLatticeBasis}
}\hfill{\scriptsize (filter)}}\\


 For \mbox{\texttt{\mdseries\slshape L}} satisfying \texttt{IsOctonionLattice} the attributes \texttt{Basis(\mbox{\texttt{\mdseries\slshape L}})} and \texttt{ CanonicalBasis(\mbox{\texttt{\mdseries\slshape L}})} are equivalent. The corresponding basis satisfies \texttt{IsOctonionLatticeBasis(\mbox{\texttt{\mdseries\slshape B}})} and provides a basis for octonion lattice \mbox{\texttt{\mdseries\slshape L}} as a left free Z-module. In turn, \texttt{BasisVectors(\mbox{\texttt{\mdseries\slshape B}})} are given by \texttt{LLLReducedBasisCoefficients(\mbox{\texttt{\mdseries\slshape L}})}. }

 }

 
\section{\textcolor{Chapter }{Octonion Lattice Operations}}\logpage{[ 5, 4, 0 ]}
\hyperdef{L}{X79F28E887AF17FFC}{}
{
  

\subsection{\textcolor{Chapter }{Rank}}
\logpage{[ 5, 4, 1 ]}\nobreak
\hyperdef{L}{X827146F37E2AA841}{}
{\noindent\textcolor{FuncColor}{$\triangleright$\enspace\texttt{Rank({\mdseries\slshape L})\index{Rank@\texttt{Rank}}
\label{Rank}
}\hfill{\scriptsize (operation)}}\\


 For \mbox{\texttt{\mdseries\slshape L}} satisfying \texttt{IsOctonionLattice} these attributes determine the lattice rank, which is equivalent to the
lattice dimension. The value is computed by determining \texttt{Rank(GeneratorsAsCoefficients(\mbox{\texttt{\mdseries\slshape L}}))}. }

 

\subsection{\textcolor{Chapter }{ScalarProduct}}
\logpage{[ 5, 4, 2 ]}\nobreak
\hyperdef{L}{X86161E22826E7C0D}{}
{\noindent\textcolor{FuncColor}{$\triangleright$\enspace\texttt{ScalarProduct({\mdseries\slshape L, x, y})\index{ScalarProduct@\texttt{ScalarProduct}}
\label{ScalarProduct}
}\hfill{\scriptsize (operation)}}\\


For \mbox{\texttt{\mdseries\slshape L}} that satisfies \texttt{IsOctonionLattice} and \mbox{\texttt{\mdseries\slshape x}}, \mbox{\texttt{\mdseries\slshape y}} either octonion vectors or coefficient vectors, this operation computes \texttt{Trace(\mbox{\texttt{\mdseries\slshape x}}*\mbox{\texttt{\mdseries\slshape g}}*ComplexConjugate(\mbox{\texttt{\mdseries\slshape y}}))}. }

 

\subsection{\textcolor{Chapter }{\texttt{\symbol{92}}in}}
\logpage{[ 5, 4, 3 ]}\nobreak
\hyperdef{L}{X87BDB89B7AAFE8AD}{}
{\noindent\textcolor{FuncColor}{$\triangleright$\enspace\texttt{\texttt{\symbol{92}}in({\mdseries\slshape x, L})\index{\texttt{\symbol{92}}in@\texttt{\texttt{\symbol{92}}in}}
\label{bSlashin}
}\hfill{\scriptsize (operation)}}\\


 For \mbox{\texttt{\mdseries\slshape x}} an octonion vector (satisfies \texttt{IsOctonionCollection} and \mbox{\texttt{\mdseries\slshape L}} an octonion lattice (satisfies \texttt{IsOctonionLattice}), this function evaluates inclusion of \mbox{\texttt{\mdseries\slshape x}} in \mbox{\texttt{\mdseries\slshape L}}. Note that \texttt{\texttt{\symbol{92}}in(\mbox{\texttt{\mdseries\slshape x}},\mbox{\texttt{\mdseries\slshape L}})} and \texttt{\mbox{\texttt{\mdseries\slshape x}} in \mbox{\texttt{\mdseries\slshape L}}} are equivalent expression. }

 
\subsection{\textcolor{Chapter }{Sublattice Identification}}\logpage{[ 5, 4, 4 ]}
\hyperdef{L}{X7F68456883DCEE5D}{}
{
\noindent\textcolor{FuncColor}{$\triangleright$\enspace\texttt{IsSublattice({\mdseries\slshape L, M})\index{IsSublattice@\texttt{IsSublattice}}
\label{IsSublattice}
}\hfill{\scriptsize (operation)}}\\
\noindent\textcolor{FuncColor}{$\triangleright$\enspace\texttt{IsSubset({\mdseries\slshape L, M})\index{IsSubset@\texttt{IsSubset}}
\label{IsSubset}
}\hfill{\scriptsize (operation)}}\\


 For both \mbox{\texttt{\mdseries\slshape L}} and \mbox{\texttt{\mdseries\slshape M}} octonion lattices (satisfies \texttt{IsOctonionLattice}) these two functions determine whether the elements of \mbox{\texttt{\mdseries\slshape M}} are contained in \mbox{\texttt{\mdseries\slshape L}}. }

 

\subsection{\textcolor{Chapter }{\texttt{\symbol{92}}=}}
\logpage{[ 5, 4, 5 ]}\nobreak
\hyperdef{L}{X806A4814806A4814}{}
{\noindent\textcolor{FuncColor}{$\triangleright$\enspace\texttt{\texttt{\symbol{92}}=({\mdseries\slshape L, M})\index{\texttt{\symbol{92}}=@\texttt{\texttt{\symbol{92}}=}}
\label{bSlash=}
}\hfill{\scriptsize (operation)}}\\


 For both \mbox{\texttt{\mdseries\slshape L}} and \mbox{\texttt{\mdseries\slshape M}} octonion lattices (satisfies \texttt{IsOctonionLattice}) the expression \texttt{\mbox{\texttt{\mdseries\slshape L}} = \mbox{\texttt{\mdseries\slshape M}}} return true when \texttt{IsSublattice(\mbox{\texttt{\mdseries\slshape L}}, \mbox{\texttt{\mdseries\slshape M}}) and IsSublattice(\mbox{\texttt{\mdseries\slshape L}}, \mbox{\texttt{\mdseries\slshape M}}) }. }

 
\subsection{\textcolor{Chapter }{Converting Between Real and Octonion Vectors}}\logpage{[ 5, 4, 6 ]}
\hyperdef{L}{X8677AEDB8631EF5E}{}
{
\noindent\textcolor{FuncColor}{$\triangleright$\enspace\texttt{RealToOctonionVector({\mdseries\slshape L, x})\index{RealToOctonionVector@\texttt{RealToOctonionVector}!RealToOctLattices}
\label{RealToOctonionVector:RealToOctLattices}
}\hfill{\scriptsize (function)}}\\
\noindent\textcolor{FuncColor}{$\triangleright$\enspace\texttt{OctonionToRealVector({\mdseries\slshape L, y})\index{OctonionToRealVector@\texttt{OctonionToRealVector}!OctToRealLattices}
\label{OctonionToRealVector:OctToRealLattices}
}\hfill{\scriptsize (function)}}\\


 Let \mbox{\texttt{\mdseries\slshape L}} be an octonion lattice, satisfying \texttt{IsOctonionLattice}, and let \mbox{\texttt{\mdseries\slshape B}} be a basis for the octonion algebra \texttt{UnderlyingOctonionRing(\mbox{\texttt{\mdseries\slshape L}})}. Let \mbox{\texttt{\mdseries\slshape x}} be a real vector with \texttt{Length(\mbox{\texttt{\mdseries\slshape x}}) mod 8 = 0} and let \mbox{\texttt{\mdseries\slshape y}} be an octonion vector of length \texttt{Dimension(\mbox{\texttt{\mdseries\slshape L}})/8}. The function \texttt{RealToOctonionVector(\mbox{\texttt{\mdseries\slshape B}}, \mbox{\texttt{\mdseries\slshape x}})} returns an octonion vector constructed by taking each successive octonion
entry as the linear combination in the eight basis vectors of \mbox{\texttt{\mdseries\slshape B}} of the corresponding eight real coefficients. Likewise, the function \texttt{OctonionToRealVector(\mbox{\texttt{\mdseries\slshape B}}, \mbox{\texttt{\mdseries\slshape y}})} is the concatenation of the real coefficients of the octonion entries computed
using the basis \mbox{\texttt{\mdseries\slshape B}}. In contrast, \texttt{RealToOctonionVector(\mbox{\texttt{\mdseries\slshape L}}, \mbox{\texttt{\mdseries\slshape x}})} returns the linear combination of the octonion lattice canonical basis vectors
defined by \texttt{LLLReducedBasisCoefficients(\mbox{\texttt{\mdseries\slshape L}})} given by the coefficients \mbox{\texttt{\mdseries\slshape x}}. The function \texttt{OctonionToRealVector(\mbox{\texttt{\mdseries\slshape L}}, \mbox{\texttt{\mdseries\slshape y}})} determines the lattice coefficients of octonion vector \mbox{\texttt{\mdseries\slshape y}} in the canonical basis of octonion lattice \mbox{\texttt{\mdseries\slshape L}}. 
\begin{Verbatim}[commandchars=!@|,fontsize=\small,frame=single,label=Example]
  !gapprompt@gap>| !gapinput@O := OctonionArithmetic(Integers); B := Basis(O);|
  <algebra of dimension 8 over Integers>
  CanonicalBasis( <algebra of dimension 8 over Integers> )
  !gapprompt@gap>| !gapinput@L := OctonionLatticeByGenerators(Concatenation(List(B, x -> x*IdentityMat(3))));|
  <free left module over Integers, with 24 generators>
  Time of last command: 464 ms
  !gapprompt@gap>| !gapinput@List(IdentityMat(24), x -> RealToOctonionVector(L, x)) = List(LLLReducedBasisCoefficien|
  ts(L), y -> RealToOctonionVector(Basis(O), y));
  true
\end{Verbatim}
 Another example illustrates the inverse properties of these functions. 
\begin{Verbatim}[commandchars=!@|,fontsize=\small,frame=single,label=Example]
  !gapprompt@gap>| !gapinput@OctonionToRealVector(L, RealToOctonionVector(L, [1..24])) = [1..24];|
  true
  !gapprompt@gap>| !gapinput@OctonionToRealVector(Basis(O),RealToOctonionVector(Basis(O), [1..24])) = [1..24];|
  true
\end{Verbatim}
 }

 }

 }

 \def\bibname{References\logpage{[ "Bib", 0, 0 ]}
\hyperdef{L}{X7A6F98FD85F02BFE}{}
}

\bibliographystyle{alpha}
\bibliography{zotero}

\addcontentsline{toc}{chapter}{References}

\def\indexname{Index\logpage{[ "Ind", 0, 0 ]}
\hyperdef{L}{X83A0356F839C696F}{}
}

\cleardoublepage
\phantomsection
\addcontentsline{toc}{chapter}{Index}


\printindex

\newpage
\immediate\write\pagenrlog{["End"], \arabic{page}];}
\immediate\closeout\pagenrlog
\end{document}
